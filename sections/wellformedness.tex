\section{Well-Formedness of (Co)Inductive Definitions}\label{sec:wf-ind}

In this section we define what it means for a \coinductive definition to be \emph{well-formed}, including some required auxilliary definitions.
A signature is then well-formed if each of its \coinductive definitions are well-formed.
Note that although we prove subject reduction for \lang without nested inductive types, we include their definitions for completeness.

\begin{definition}[Strict Positivity]
  Given some existing sigature $\Sigma$, the variable $x$ occurs \emph{strictly positively} in the term $t$, written $x \POS t$, if any of the following holds:

  \begin{itemize}
    \item $x \notin \FV(t)$
    \item $t \conv x ~ \vec{e}$ and $x \notin \FV(\vec{e})$
    \item $t \conv \prod{x}{u}{v}$ and $x \notin \FV(u)$ and $x \POS v$
  \end{itemize}

  If nested inductive types are permitted, then $x \POS t$ may hold if the following also holds:

  \begin{itemize}
    \item $t \conv I_k^\infty ~ \vec{p} ~ \vec{a}$ where
      $\defn{\langle \app{I_i}{\Delta_p}}{\any \rangle}{\seq{\assm{c_j}{\prodctx{\Delta_j}{\app{I_j}{\dom{\Delta_p}}{\vec{t}_j}}}}} \in \Sigma$
      for some $k \in \vec{\imath}$ and all of the following hold:
      \begin{itemize}
        \item $\norm{\vec{p}} = \norm{\Delta_p}$
        \item $x \notin \FV(\vec{a})$
        \item For every $j$, if $I_j = I_k$, then $x \nPOS_{I_k} \subst{(\prodctx{\Delta_j}{\app{I_j}{\vec{p}}{\vec{t}_j}})}{\dom{\Delta_p}}{\vec{p}}$
      \end{itemize}
  \end{itemize}
\end{definition}

\begin{definition}[Nested Positivity]
  Given some existing signature $\Sigma$, the variable $x$ is \emph{nested positive} in $t$ of $I_k$, written $x \nPOS_{I_k} t$, if
  $\defn{\langle \app{I_i}{\Delta_p}}{\any \rangle}{\any} \in \Sigma$
  for some $k \in \vec{\imath}$ and any of the following holds:

  \begin{itemize}
    \item $t \conv \app{I_k^\infty}{\vec{p}}{\vec{a}}$ and $\norm{\vec{p}} = \norm{\Delta_p}$ and $x \notin \FV(\vec{a})$
    \item $t \conv \prod{x}{u}{v}$ and $x \POS u$ and $x \nPOS_{I_k} v$
  \end{itemize}

  In short, $x \nPOS_I t$ if $t \conv \prodctx{\Delta}{\app{I}{\vec{p}}{\vec{a}}}$ and $x \POS \Delta$ and $x \notin \FV(\vec{a})$.
\end{definition}

\begin{definition}[Constructor Type]
  The term $t$ is a constructor type for $I$ when:

  \begin{itemize}
    \item $t = I ~ \vec{e}$; or
    \item $t = \prod{x}{u}{v}$ where $x \notin \FV{u}$ and $v$ is a constructor type for $I$; or
    \item $t = u \to v$ where $x \POS u$ and $v$ is a constructor type for $I$.
  \end{itemize}

  Note that in particular, this means that $t = \prodctx{\Delta}{I ~ \vec{e}}$ such that $I \POS u$
  for every $u \in \codom{\Delta}$, and the recursive arguments of $t$ are not dependent.
\end{definition}

\begin{definition}[Well-formedness of \Coinductive Definitions]
  Suppose we have some signature $\Sigma$ and some global environment $\Gamma_G$. Consider the following \coinductive definition, where $\vec{p} = \dom{\Delta_p}$.
  \begin{displaymath}
    \defn{\langle \app{I_i}{\Delta_p}}{\prodctx{\Delta_i}{U_i} \rangle}{\seq{\assm{c_j}{\prodctx{\Delta_j}{\app{I_j}{\vec{p}}{\vec{t}_j}}}}}
  \end{displaymath}
  This \coinductive definition is \emph{well-formed} if the following all hold:

  \begin{enumerate}[label = \textbf{(I\arabic*)}.]
    \item For every $i$, there is some $U'_i$ such that $\Sigma, \Gamma_G, \Delta_p \vdash \prodctx{\Delta_i}{U_i} : U'_i$ holds.
    \item For every $j$, there is some $U_j$ such that $\Sigma, \Gamma_G, \Delta_p(I_j^\infty: \prodctx{\Delta_p}{\prodctx{\Delta_i}{U_i}}) \vdash \prodctx{\Delta_j}{I_j^\infty ~ \vec{p} ~ \vec{t}_j} : U_j$ holds.
    \item For every $j$, $\prodctx{\Delta_j}{\app{I_j}{\vec{p}}{\vec{t}_j}}$ is a constructor type for $I_j$. Note that this implies $I_j \POS \codom{\Delta_j}$.
    \item For every $i, j$, all \coinductive types in the terms $\codom{\Delta_p}, \codom{\Delta_i}, \codom{\Delta_j}$ are annotated with $\infty$.
  \end{enumerate}
\end{definition}
