\section{\titlelang}\label{sec:typing}
In this section, we introduce the syntax and judgements of \lang,
culminating in the typing and well-formedness judgements.
Note that this is the core calculus, which is produced from plain CIC by the inference algorithm,
introduced in \autoref{sec:algorithm}.

\subsection{Syntax}

\begin{figure}
\centering

\begin{align*}
m, n, i, j, k, \ell &\Coloneqq \meta{positive naturals} \\
f, g, h, x, y, z &\Coloneqq \meta{term variables} &
\tau, \upsilon &\Coloneqq \meta{size variables} \\
I &\Coloneqq \meta{(co)inductive type names} &
c &\Coloneqq \meta{constructor names} \\
r, s &\Coloneqq \new{\upsilon \mid \hat{s} \mid \infty} \qquad
\rho \Coloneqq \new{\set{\vec{\upsilon \mapsto s}}} &
U &\Coloneqq \Prop \mid \Set \mid \Type{n} \\
e, a, b, p, q, t, u, v, P &\Coloneqq
  \mathrlap{x
  \mid \new{x^{\rho}}
  \mid U
  \mid \prod{x}{t}{t}
  \mid \abs{x}{t^\circ}{e}
  \mid \app{e}{e}
  \mid \letin{x}{t^\circ}{e}{e}
  \mid \new{I^s}
  \mid c} \\
&\mathrlap{\mid \caseof{P^\circ}{e}{c}{e}
  \mid \fix{m}{f^n}{\new{t^*}}{e}
  \mid \cofix{m}{f}{\new{t^*}}{e}} \\
\dots, e^\circ, t^\circ, P^\circ &\Coloneqq
  \mathrlap{x
  \mid U
  \mid \prod{x}{t^\circ}{t^\circ}
  \mid \abs{x}{t^\circ}{e^\circ}
  \mid \app{e^\circ}{e^\circ}
  \mid \letin{x}{t^\circ}{e^\circ}{e^\circ}
  \mid I
  \mid c} \\
&\mathrlap{\mid \caseof{P^\circ}{e^\circ}{c}{e^\circ}
  \mid \fix{m}{f^n}{t^\circ}{e^\circ}
  \mid \cofix{m}{f}{t^\circ}{e^\circ}}
\end{align*}

\begin{align*}
\Delta &\Coloneqq \mt \mid \Delta(\assm{x}{e}) &\textit{telescopes} \\
\Gamma &\Coloneqq \mt \mid \Gamma(\assm{x}{e}) \mid \Gamma(\defn{x}{t}{t}) &\textit{local environments} \\
\Gamma_G &\Coloneqq \mt \mid \Gamma_G(\Assm{x}{e}) \mid \Gamma_G(\Defn{x}{t}{t}) &\textit{global environments} \\
\Sigma &\Coloneqq \mt \mid \Sigma(\seq{\vec{\assm{\app{I_i}{\Delta_p}}{\prodctx{\new{\Delta^\infty_i}}{U_i}}}} \coloneqq \seq{\vec{\assm{c_j}{\prodctx{\new{\Delta^\infty_j}}{\app{I_j}{\dom{\Delta_p}}{\new{\overline{t^\infty}_j}}}}}}) &\textit{signatures} \\
\end{align*}

\caption{Syntax of \lang terms, environments, and signatures}
\label{fig:syntax}
\end{figure}

%%% Local Variables:
%%% TeX-master: "../main.tex"
%%% TeX-engine: default
%%% End:
\label{sec:typing:syntax}

The syntax of \lang, environments, and signatures are described in \autoref{fig:syntax}.
It is a standard CIC with expressions (or terms) consisting of cumulative universes, dependent functions, definitions, \coinductives, case expressions, and mutual \cofixpoints.
Additions relevant to sized types are highlighted in grey,
which we explain in detail shortly.
Notation such as syntactic sugar or metafunctions and metarelations will also be highlighted in grey
where they are first introduced in the prose.

The brackets $\seq{\ph}$ denotes a sequence of syntactic constructions.
We use 1-based indexing for sequences using subscripts;
sequences only range over a single index unless otherwise specified.
Ellipses may be used in place of the brackets where it is clearer;
for instance, the branches of a case expression are written as
$\langle \vec{c_j \Rightarrow e_j} \rangle$ or
$\langle c_1 \Rightarrow e_1, \dots, c_j \Rightarrow e_j, \dots \rangle$,
and $e_j$ is the $j$th branch expression in the sequence.
Additionally, $\new{\app{e}{\vec{a}}}$ is syntactic sugar for application of $e$ to the terms in $\vec{a}$.

\subsubsection{Size Annotations and Substitutions}

As we have seen, \coinductive types are annotated with a size expression representing its size.
A \coinductive with an \textit{infinite} $\infty$ size annotation is said to be a \textit{full type},
representing \coinductives of all sizes.
Otherwise, an inductive with a \textit{noninfinite} size annotation $s$ represents inductives of size $s$ or \emph{smaller},
while a coinductive with annotation $s$ represents coinductives of size $s$ or \emph{larger}.
This captures the idea that an object of an inductive type has some amount of content to be consumed,
while one of a coinductive type must produce some amount of content.

As a concrete example, a list with $s$ elements has type $\app{\List^{s}}{t}$, because it has at most $s$ elements,
but it also has type $\app{\List^{\succ{s}}}{t}$, necessarily having at most $\succ{s}$ elements as well.
On the other hand, a stream producing at least $\succ{s}$ elements has type $\app{\Stream^{\succ{s}}}{t}$,
and also has type $\app{\Stream^{s}}{t}$ since it necessarily produces at least $s$ elements as well.
These ideas are formalized in the subtyping rules in an upcoming subsection.

Variables bound by local definitions (introduced by let expressions) and constants bound by global definitions (introduced in global environments)
are annotated with a \textit{size substitution} that maps size variables to size expressions.
The substitutions are performed during their reduction.
As mentioned in the previous section, this makes definitions size polymorphic.

In the type annotations of functions and let expressions, as well as the motive of case expressions,
rather than ordinary \textit{sized terms}, we instead have \textit{bare terms} $t^\circ$.
This denotes terms where size annotations are removed.
These terms are required to be bare in order to preserve subject reduction without requiring explicit size applications or typed reduction,
both of which would violate backward compatibility with Coq.
We give an example of the loss of subject reduction when type annotations are not bare in \autoref{sec:metatheory:sr:bare}
% \citet{cic-hat-minus} discusses the issue in further detail.
Similarly, we use $t^\infty$ to denote \textit{full terms}, which appear in signatures.

\subsubsection{Fixpoints and Cofixpoints}

In contrast to \CIChat and \CIChatminus, \lang has mutual \cofixpoints.
In a mutual fixpoint $\fix{m}{f_k^{n_k}}{t_k^*}{e_k}$, each $\defn{f_k^{n_k}}{t_k^*}{e_k}$ is one fixpoint definition.
$n_k$ is the index of the recursive argument of $f_k$, and $\kw{fix}_m$ means that the $m$th fixpoint definition is selected.
Instead of bare terms, fixpoint type annotations are \textit{position terms} $t^*$,
where size annotations are either removed or replaced with a \textit{position annotation} $\ast$.
They occur on the inductive type of the recursive argument, as well as the return type if it is an inductive with the same or smaller size.
For instance (using $\new{t \rightarrow u}$ as syntactic sugar for $\prod{\any}{t}{u}$),
the recursive function $\fix*{1}{\mathit{minus}^1 : \Nat^* \rightarrow \Nat \rightarrow \Nat^* \coloneqq \dots}$
has a position-annotated return type since the return value will not be any larger than that of the first argument.

Mutual cofixpoints $\cofix{m}{f_k}{t_k^*}{e_k}$ are similar, except cofixpoint definitions don't need $n_k$,
as cofixpoints corecursively produce a coinductive rather than recursively consuming an inductive.
Position annotations occur on the coinductive return type as well as any coinductive argument types with the same size or smaller.
As an example, $\cofix*{1}{\mathit{dup} : \prod{A}{\Set}{\app{\Stream^*}{A} \rightarrow \app{\Stream^*}{A}} \coloneqq \dots}$,
a corecursive function that duplicates each element of a stream,
has a position-annotated argument type since it returns a larger stream.

Position annotations mark the size annotation locations in the type of the \cofixpoint where we are allowed to assign the \emph{same} size expression.
This is why we can give the $\mathit{minus}$ fixpoint the type $\Nat^{\succ{\upsilon}} \rightarrow \Nat^\infty \rightarrow \Nat^{\succ{\upsilon}}$, for instance.
In general, if a \cofixpoint has a position annotation on an argument type and the return type,
we say that it is \textit{size-preserving} in that argument.
Inuitively, $f$ is size-preserving over an argument $e$ if using $\app{f}{e}$ in place of $e$ should be allowed, size-wise.

\subsubsection{Environments and Signatures}

We divide environments into local and global ones.
They consist of \textit{declarations}, which can be either \textit{assumptions} or \textit{definitions}.
While local environments represent bindings introduced by functions and let expressions,
global environments represent top-level declarations corresponding to Coq vernacular.
We may also refer to global environments alone as \textit{programs}.
Telescopes (that is, environments consisting only of local assumptions) are used in syntactic sugar:
given $\Delta = \vec{(\assm{x_i}{t_i})}$, $\new{\prodctx{\Delta}{t}}$ is sugar for $\prod{x_1}{t_1}{\dots \prod{x_i}{t_i}{\dots t}}$, while $\new{\dom{\Delta}}$ is the sequence $\vec{x_i}$.
Additionally, $\Delta^\infty$ denotes telescopes containing only full terms.

We use $\new{x \in \Gamma}$, $\new{(\assm{x}{t}) \in \Gamma}$, and $\new{(\defn{x}{t}{e}) \in \Gamma}$
to represent the presence of some declaration binding $x$, the given assumption, and the given definition in $\Gamma$, respectively,
and similarly for $\Gamma_G$ and $\Delta$.

Signatures consist of mutual \coinductive definitions.
For simplicity, throughout the judgements in this paper, we assume some fixed, implicit signature $\Sigma$.
Global environments and signatures should be easily extendible to an interleaving of declarations and \coinductive definitions,
which would be more representative of a real program.
A mutual \coinductive definition
$\seq{\vec{\assm{\app{I_i}{\Delta_p}}{\prodctx{\Delta^\infty_i}{U_i}}}} \coloneqq
\seq{\vec{\assm{c_j}{\prodctx{\Delta^\infty_j}{\app{I_j}{\dom{\Delta_p}}{\vec{t^\infty}_j}}}}}$
consists of the following:
\begin{itemize}
  \item $I_i$, the names of the defined \coinductive types;
  \item $\Delta_p$, the \textit{parameters} common to all $I_i$;
  \item $\Delta^\infty_i$, the \textit{indices} of each $I_i$;
  \item $U_i$, the universe to which $I_i$ belongs;
  \item $c_j$, the names of the defined constructors;
  \item $\Delta^\infty_j$, the arguments of each $c_j$;
  \item $I_j$, the \coinductive type to which $c_j$ belongs; and
  \item $\vec{t^\infty}_j$, the indices to $I_j$.
\end{itemize}

We require that the index and argument types be \emph{full} types and terms.
Note also that $I_j$ is the \coinductive type of the $j$th constructor, \emph{not} the $j$th \coinductive in the sequence $I_i$.
We forgo the more precise notation $I_{i_j}$ for brevity.

As a concrete example, the usual $\Vector$ type (using $\new{(\assm{x}{t}) \rightarrow u}$ as syntactic sugar for $\prod{x}{t}{u}$) would be defined as:
\begin{align*}
  \seq{\assm{\app{\Vector}{(A : \Type{})}&}{\Nat \to \Type{}}} \coloneqq \\
      \seq{\assm{\VNil&}{\app{\Vector}{A}{\Zero}}, \\
      \assm{\VCons&}{(\assm{n}{\Nat}) \to A \to \app{\Vector}{A}{(\app{\Succ}{n}))}}}.
\end{align*}

As mentioned in the previous section, unlike \CIChat and \CIChatminus, our \coinductive definitions do not have parameter polarity annotations.
In those languages, for $\Vector$'s parameter for instance, they might write $(\assm{A^+}{\Type{}})$, giving it positive polarity, so that 
$\app{\Vector^\infty}{\Nat^s}{n}$ is a subtype of $\app{\Vector^\infty}{\Nat^{\succ{s}}}{n}$.

As is standard, the well-formedness of \coinductive definitions depends not only on the well-typedness of its types but also on syntactic positivity conditions.
We reproduce the \textit{strict positivity} conditions in \autoref{sec:wf-ind}, and refer the reader to clauses I1--I9 in \citet{cic-hat-minus}, clauses 1--7 in \mbox{\citet{cic-hat}}, and the Coq Manual~\citep{coq} for further details.
As \lang does not support nested \coinductives,
we don't need the corresponding notion of \textit{nested positivity}.
Furthermore, we assume that our fixed, implicit signature is well-formed.

\iffalse
\subsubsection{Metafunctions and Metarelations}

We declare the following metafunctions, whose definitions are straightforward:

\begin{itemize}
    \item $\FV{\ph}, \SV{\ph}$ return the set of free term and size variables in a given term or environment, respectively;
    \item $\floor{\ph}$ returns the size variable a given finite (\ie not $\infty$) size expression;
    \item $\norm{\ph}$ returns the cardinality of its argument (\eg sequence length, set size, \etc);
    \item $\erase{\ph}$ erases sized terms to bare terms; and
    \item $\erase{\ph}^\upsilon$ replaces size annotations with size variable $\upsilon$ by $*$ and removes all other ones.
\end{itemize}

Note that the free size variables of a size substition $\rho$ are only those of the sizes that $\rho$ maps \emph{to}.
\fi

\subsection{Reduction and Convertibility}

The reduction rules listed in \autoref{fig:reduction} are the usual ones for CIC with definitions:
$\beta$\=/reduction (function application),
$\zeta$\=/reduction (let expression evaluation),
$\iota$\=/reduction (case expressions),
$\mu$\=/reduction (fixpoint expressions),
$\nu$\=/reduction (cofixpoint expressions),
$\delta$\=/reduction (local definitions), and
$\Delta$\=/reduction (global definitions).

\begin{figure}
\fbox{$\gg \vdash e \reduce_{\beta\zeta\delta\Delta\iota\mu\nu} e$} \hfill
\begin{align*}
\gg \vdash x^\rho & \reduce_\delta \rho e \qquad \textit{where} ~ (x : t \coloneqq e) \in \Gamma \\
\gg \vdash x^\rho & \reduce_\Delta \rho e \qquad \textit{where} ~ (\Defn{x}{t}{e}) \in \Gamma_G \\
\gg \vdash \app{(\abs{x}{t^\circ}{e_1})}{e_2} & \reduce_\beta \subst{e_1}{x}{e_2} \\
\gg \vdash \letin{x}{t^\circ}{e_1}{e_2} & \reduce_\zeta \subst{e_2}{x}{e_1} \\
\gg \vdash \caseof{P^\circ}{(\app{c_\ell}{\vec{p}}{\vec{a}})}{c_j}{e_j} & \reduce_\iota \app{e_\ell}{\vec{a}} \\
\gg \vdash \app{q_m}{\vec{b}}{(\app{c_\ell}{\vec{p}}{\vec{a}})}
  & \reduce_\mu \app{\substvec{e_m}{f_k}{q_k}}{\vec{b}}{(\app{c_\ell}{\vec{p}}{\vec{a}})} \\
  \textit{where} ~ & \forall i \in \vec{k}, q_i \equiv \fix{i}{f_k^{n_k}}{t_k}{e_k}, \norm{\vec{b}} = n_m - 1 \\
\gg \vdash \caseof{P^\circ}{(\app{q_m}{\vec{b}})}{c_j}{a_j}
  & \reduce_\nu \caseof{P^\circ}{(\app{\substvec{e_m}{f_k}{q_k}}{\vec{b}})}{c_j}{a_j} \\
  \textit{where} ~ & \forall i \in \vec{k}, q_i \equiv \cofix{i}{f_k}{t_k}{e_k}
\end{align*}
\caption{Reduction rules}
\label{fig:reduction}
\end{figure}

\begin{figure}
\fbox{$\gg \vdash e \reduce^* e$} \hfill
\vspace{-2ex}

\begin{mathpar}
  \inferrule[\defrule{red-refl}]{~}{
    \gg \vdash e \reduce^* e
  }
  \and
  \inferrule[\defrule{red-trans}]{
    \gg \vdash e_1 \reduce e_2 \and
    \gg \vdash e_2 \reduce^* e_3
  }{
    \gg \vdash e_1 \reduce^* e_3
  }
\end{mathpar}
\caption{Multi-step reduction rules}
\label{fig:reductions}
\end{figure}

\begin{figure}
\fbox{$\gg \vdash e \conv e$} \hfill
\vspace{-3ex}

\begin{mathpar}
  \inferrule[\defrule{conv-cong}]
    {\text{For every $i$:} \and
      \gg \vdash a_i \conv b_i}
    {\gg \vdash \substvec{e}{x_i}{a_i} \conv \substvec{e}{x_i}{b_i}}
  \and
  \inferrule[\defrule{conv-red}]
    {\gg \vdash e_1 \reduce^* e \\\\
      \gg \vdash e_2 \reduce^* e}
    {\gg \vdash e_1 \conv e_2}
  \\
  \inferrule[\defnamerule{conv-eta-r}{conv-$\eta$-l}]
    {\gg \vdash e_1 \reduce^* \abs{x}{\erase{t}}{e} \\\\
      \gg \vdash e_2 \reduce^* e'_2 \\\\
      \gg (x:t) \vdash e \conv e'_2 ~ x}
    {\gg \vdash e_1 \conv e_2}
  \and
  \inferrule[\defnamerule{conv-eta-1}{conv-$\eta$-r}]
    {\gg \vdash e_1 \reduce^* e'_1 \\\\
      \gg \vdash e_2 \reduce^* \abs{x}{\erase{t}}{e} \\\\
      \gg (x:t) \vdash e'_1 ~ x \conv e}
    {\gg \vdash e_1 \conv e_2}
\end{mathpar}
\caption{Convertibility rules}
\label{fig:convertibility}
\end{figure}

%%% Local Variables:
%%% TeX-master: "../main.tex"
%%% TeX-engine: default
%%% End:


In the case of \deltaDeltareduction, where the variable or constant has a size substitution annotation, we modify the usual rules.
These reduction rules are important for supporting size inference with definitions.
If the definition body contains \coinductive types (or other defined variables and constants), we can assign them fresh size variables for each distinct usage of the defined variable.
Further details are discussed in \autoref{sec:algorithm}.

Much of the reduction behaviour is expressed in terms of term and size substitution.
Capture-avoiding substitution is denoted with $\new{\subst{e}{x}{e'}}$,
and simultaneous substitution with $\new{\substvec{e}{x_i}{e_i}}$.
$\new{\rho e}$ denotes applying the substitutions $\substvec{e}{\upsilon_i}{s_i}$ for every $\upsilon_i \mapsto s_i$ in $\rho$,
and similarly for $\new{\rho s}$.

This leaves applications of size substitutions to environments,
and to size substitutions themselves when they appear as annotations on variables and constants.
A variable $x^{\set{\vec{\upsilon \mapsto s}}}$ bound to $\defn{x}{t}{e}$ in the environment, for instance,
can be thought of as a delayed application of the sizes $\vec{s}$,
with the definition implicitly abstracting over all size variables $\vec{\upsilon}$.
Therefore, the ``free size variables'' of the annotated variable are those in $\vec{s}$,
and given some size substitution $\rho$,
$\rho x^{\set{\vec{\upsilon \mapsto s}}} = x^{\set{\vec{\upsilon \mapsto \rho s}}}$.
Meanwhile, we treat all $\vec{\upsilon}$ in the definition as \emph{bound},
so that $\rho(\Gamma_1 \defn{x}{t}{e} \Gamma_2) = (\rho\Gamma_1)(\defn{x}{t}{e})(\rho\Gamma_2)$,
skipping over all definitions, and similarly for global environments.

Finally, $\new{\ph \equiv \ph}$ is syntactic equality up to $\alpha$-equivalence (renaming),
and $\new{\norm{\ph}}$ yields the cardinality of its argument (\eg sequence length, set size, \etc).

We define reduction ($\reduce$) as the congruent closure of the reductions,
multi-step reduction ($\rhd^*$) in \autoref{fig:reductions} as the reflexive--transitive closure of $\rhd$,
and convertibility ($\conv*$) in \autoref{fig:convertibility}.
The latter also includes $\eta$-convertibility,
which is presented informally in the Coq manual~\citep{coq} and formally (but part of typed conversion) in \citet{conversion}.
Note that there are no explicit rules for symmetry and transitivity of convertibility
because these properties are derivable, as proven in\opcitt{conversion}

\iffalse
We also use the metafunction \whnf to denote the reduction of a term to weak head normal form,
which would have the form of a universe, a function type, an unapplied function,
a \coinductive type (applied or unapplied), a constructor (applied or unapplied),
an assumption variable (applied or unapplied), or an unapplied \cofixpoint,
with arguments and inner terms unreduced.
\fi

\subsection{Subtyping and Positivity}\label{subsec:typing:subtyping}

\begin{figure}
\fbox{$s \sqsubseteq s$} \hfill
\begin{mathpar}
  \inferrule[\defrule{ss-infty}]{~}
    {s \sqsubseteq \infty}
  \and
  \inferrule[\defrule{ss-refl}]{~}
    {s \sqsubseteq s}
  \and
  \inferrule[\defrule{ss-succ}]{~}
    {s \sqsubseteq \hat{s}}
  \and
  \inferrule[\defrule{ss-trans}]
    {s_1 \sqsubseteq s_2 \and s_2 \sqsubseteq s_3}
    {s_1 \sqsubseteq s_3}
\end{mathpar}
\caption{Subsizing rules}
\label{fig:subsizing}
\end{figure}

%%% Local Variables:
%%% TeX-master: "../main.tex"
%%% TeX-engine: default
%%% End:


First, we define the subsizing relation in \autoref{fig:subsizing}.
Subsizing is straightforward since our size algebra is simple.
Notice that both $\infty \sqsubseteq \succ{\infty}$ and $\succ{\infty} \sqsubseteq \infty$ hold.

\begin{figure}
\begin{flushleft}
  \fbox{$\gg \vdash t \leq t$}
\end{flushleft}
\begin{mathpar}
  \inferrule[\defrule{st-cumul}]{~}
    {\gg \vdash \Prop \leq \Set \leq \Type{1}}
  \quad
  \inferrule{~}
    {\gg \vdash \Type{i} \leq \Type{i+1}}
  \and
  \inferrule[\defrule{st-conv}]
    {\gg \vdash t \approx u}
    {\gg \vdash t \leq u}
  \and
  \inferrule[\defrule{st-trans}]
    {\gg \vdash t \leq u \\\\ \gg \vdash u \leq v}
    {\gg \vdash t \leq v}
  \and
  \inferrule[\defrule{st-prod}]
    {\gg \vdash t_1 \approx t_2 \\\\ \gg(\assm{y}{t_2}) \vdash \subst{u_1}{x}{y} \leq u_2} % t_2 \leq t_1
    {\gg \vdash \prod{x}{t_1}{u_1} \leq \prod{y}{t_2}{u_2}}
  \and
  \inferrule[\defrule{st-app}]
    {\gg \vdash t_1 \leq t_2 \\\\ \gg \vdash u_1 \approx u_2} % \u_2 \leq u_1
    {\gg \vdash t_1 ~ u_1 \leq t_2 ~ u_2}
  \and
  \inferrule[\defrule{st-ind}]
    {I \textrm{ inductive } \and s \sqsubseteq s'}
    {\gg \vdash I^s \leq I^{s'}}
  \and
  \inferrule[\defrule{st-coind}]
    {I \textrm{ coinductive } \and s' \sqsubseteq s}
    {\gg \vdash I^s \leq I^{s'}}
\end{mathpar}
\caption{Subtyping rules}
\label{fig:subtyping}
\end{figure}


The subtyping rules in \autoref{fig:subtyping} extend those of cumulative CIC with rules for sized \coinductive types.
In other words, they extend those of \CIChat, \CIChatminus, and \CChatomega with universe cumulativity (and a $\Prop$ universe).
Inductive types are \emph{covariant} in their size annotations with respect to subsizing (\refrule{st-ind}),
while coinductive types are \emph{contravariant} (\refrule{st-coind}).
Coming back to the examples in the previous section, this means that
$\app{\List^{s}}{t} \leq \app{\List^{\succ{s}}}{t}$ holds as we expect,
since a list with $s$ elements has no more than $\succ{s}$ elements;
dually, $\app{\Stream^{\succ{s}}}{t} \leq \app{\Stream^{s}}{t}$ holds as well,
since a stream producing $\succ{s}$ elements also produces no fewer than $s$ elements.

Rules \refnorule{st-prod} and \refnorule{st-app} differ from past work in their variance, but correspond to those in Coq.
As \coinductive definitions have no polarity annotations,
we treat all parameters as ordinary, invariant function arguments.
The remaining rules are otherwise standard.

\begin{figure}
\centering
\begin{flushleft}
  \fbox{$\gg \vdash \upsilon \pos t$} \quad
  \fbox{$\gg \vdash \upsilon \neg t$}
\end{flushleft}
\begin{mathpar}
  \inferrule[\defnamerule{pos-neg-notin}{pos-neg-$\notin$}]
    {\gg \vdash \upsilon \notin \SV{t}}
    {\gg \vdash \upsilon \pos t \\\\ \gg \vdash \upsilon \neg t}

  \inferrule[\defrule{pos-red}]
    {\gg \vdash t \rhd^* t' \\\\
      \gg \vdash \upsilon \pos t'}
    {\gg \vdash \upsilon \pos t}

  \inferrule[\defrule{neg-red}]
    {\gg \vdash t \rhd^* t' \\\\
      \gg \vdash \upsilon \neg t'}
    {\gg \vdash \upsilon \neg t}
  
  \inferrule[\defnamerule{pos-prod}{pos-$\Pi$}]
    {\upsilon \pos u \\\\
      \gg \vdash \upsilon \notin \SV{t}}
    {\gg \vdash \upsilon \pos \prod{x}{t}{u}}

  \inferrule[\defnamerule{neg-prod}{neg-$\Pi$}]
    {\upsilon \neg u \\\\
      \gg \vdash \upsilon \notin \SV{t}}
    {\gg \vdash \upsilon \neg \prod{x}{t}{u}}

  \inferrule[\defrule{pos-ind}]
    {I ~ \text{inductive} \\\\
      \gg \vdash \upsilon \notin \SV{\overline{a}}}
    {\gg \vdash \upsilon \pos \app{I^s}{\overline{a}}}

  \inferrule[\defrule{neg-coind}]
    {I ~ \text{coinductive} \\\\
      \gg \vdash \upsilon \notin \SV{\overline{a}}}
    {\gg \vdash \upsilon \neg \app{I^s}{\overline{a}}}
\end{mathpar}
\caption{Positivity/negativity of size variables in terms}
\label{fig:posneg}
\end{figure}

%%% Local Variables:
%%% TeX-master: "../main.tex"
%%% TeX-engine: default
%%% End:


In addition to subtyping, we define a \textit{positivity} and \textit{negativity} judgements like in past work.
They are syntactic approximations of monotonicity properties of subtyping with respect to size variables;
we have that
$\upsilon \pos t \Leftrightarrow t \leq \subst{t}{\upsilon}{\hat{\upsilon}}$ and
$\upsilon \neg t \Leftrightarrow \subst{t}{\upsilon}{\hat{\upsilon}} \leq t$ hold.
Positivity and negativity are then used to indicate where position annotations are allowed to appear in the types of \cofixpoints,
as we will see in the typing rules.

\subsection{Typing and Well-Formedness}\label{sec:typing:rules}

\begin{figure}
\begin{mathpar}
  \inferrule*[right=\defrule{wf-nil}]{\Sigma ~ \text{is well-formed}}
    {\WF(\Sigma, \square, \square)}
  \\
  \inferrule*[right=\defrule{wf-local-assum}]
    {\sgg \vdash t: \varw \and x \notin \Gamma}
    {\WF{\Sigma, \Gamma_G, \Gamma (x:t)}}

  \inferrule*[right=\defrule{wf-global-assum}]
    {\Sigma, \Gamma_G, \square \vdash t: \varw \and x \notin \Gamma_G}
    {\WF{\Sigma, \Gamma_G (\Assum{x}{|t|^\infty\!\!}), \square}}
  \\
  \inferrule*[right=\defrule{wf-local-def}]
    {\sgg \vdash e: t \and x \notin \Gamma}
    {\WF{\Sigma, \Gamma_G, \Gamma (x:t \coloneqq e)}}

  \inferrule*[right=\defrule{wf-global-def}]
    {\Sigma, \Gamma_G, \square \vdash e: t \and x \notin \Gamma_G}
    {\WF{\Sigma, \Gamma_G (\Def{x}{t}{e}), \square}}
\end{mathpar}
\caption{Well-formedness of environments}
\label{fig:wf}
\end{figure}


We begin with the rules for well-formedness of local and global environments, presented in \autoref{fig:wf}.
As mentioned, this and the typing judgements implicitly contain a signature $\Sigma$, whose well-formedness is assumed.
Additionally, we use $\any$ to omit irrelevant constructions for readability.

\begin{figure}
\centering
\begin{align*}
\Axioms
    &= \set{(\Prop, \Type{1}), (\Set, \Type{1}), (\Type{i}, \Type{i+1})} \\
\Rules
    &= \set{(U, \Prop, \Prop)}
    \cup \set{(U, \Set, \Set) : U \in \set{\Prop, \Set}} \\
    &\cup \set{(\Type{i}, \Type{j}, \Type{k}) : k = \meta{max}(i, j)} \\
\Elims
    &= \set{(U_i, U, I_i) : U_i \in \set{\Set, \Type{}}}
    \cup \set{(\Prop, \Prop, I_i)} \\
    &\cup \set{(\Prop, U, I_i) : \text{$I_i$ empty or singleton}}
\end{align*}
\caption{Universe relations: Axioms, Rules, and Eliminations}
\label{fig:axruel}
\end{figure}

%%% Local Variables:
%%% TeX-master: "../main.tex"
%%% TeX-engine: default
%%% End:


\begin{figure}
\centering

\begin{align*}
    \indtype{I_k} &=
        \prodctx{\Delta_p}{\prodctx{\Delta_k}{U_k}} \\
    \constrtype{c_\ell}{s} &=
        \prodctx{\Delta_p}{\prodctx{\Delta_\ell [I_\ell^\infty \coloneqq I_\ell^s]}{I_{\ell}^{\hat{s}} ~ \dom{\Delta_p} ~ \overline{t}_\ell}} \\
    \motivetype{\overline{p}}{U}{I_k^s} &=
        \prodctx{\Delta_k[\dom{\Delta_p} \coloneqq \overline{p}]}{\prod{\any}{I_k^s ~ \overline{p} ~ \dom{\Delta_k}}{U}} \\
    \branchtype{\overline{p}}{c_\ell}{s}{\wp} &=
        \prodctx{\Delta_\ell [I_\ell^\infty \coloneqq I_\ell^s][\dom{\Delta_p} \coloneqq \overline{p}]}{\wp ~ \overline{t}_\ell[\dom{\Delta_p}\coloneqq\overline{p}] ~ (c_\ell ~ \overline{p} ~ \dom{\Delta_\ell})}
\end{align*}
\begin{displaymath}
    \textit{where}\;\:
    k \in \overline{\imath}, \ell \in \overline{\jmath},
    \big(\seq{\vec{\assm{\app{I_i}{\Delta_p}}{\prodctx{\Delta_i}{U_i}}}} \coloneqq \seq{\vec{\assm{c_j}{\prodctx{\Delta_j}{\app{I_j}{\_}{\overline{t}_j}}}}}\big) \in \Sigma
\end{displaymath}

\caption{Metafunctions for typing rules}
\label{fig:metafunctions}
\end{figure}

%%% Local Variables:
%%% TeX-master: "../main.tex"
%%% TeX-engine: default
%%% End:


\begin{figure*}[!tp]
\centering
\small

\begin{flushleft}
\fbox{$\sgg \vdash e : t$}
\end{flushleft}

\vspace{-2ex}

\begin{mathpar}
  \inferrule[\defrule{var-assum}]
    {\WF{\sgg} \and (x:t) \in \Gamma}
    {\sgg \vdash x:t}

  \inferrule[\defrule{var-def}]
    {\WF{\sgg} \and
      (x:t \coloneqq e) \in \Gamma \\\\
      \SV(e, t) = \vec{\upsilon_i} \and
      \sgg \vdash e \overline{[\upsilon_i \coloneqq s_i]} : t \overline{[\upsilon_i \coloneqq s_i]}}
    {\sgg \vdash x^{\seq{s_i}} : t \overline{[\upsilon_i \coloneqq s_i]}}

  \inferrule[\defrule{const-assum}]
    {\WF{\sgg} \and (\Assm{x}{t}) \in \Gamma_G}
    {\sgg \vdash x:t}

  \inferrule[\defrule{const-def}]
    {\WF{\sgg} \and
      (\Defn{x}{t}{e}) \in \Gamma_G \\\\
      \SV(e, t) = \vec{\upsilon_i} \and
      \Sigma, \Gamma_G, \mt \vdash e \overline{[\upsilon_i \coloneqq s_i]} : t \overline{[\upsilon_i \coloneqq s_i]}}
    {\sgg \vdash x^{\seq{s_i}} : t \overline{[\upsilon_i \coloneqq s_i]}}

  \inferrule[\defrule{univ}]
    {\WF{\sgg} \and (U_1, U_2) \in \Axioms}
    {\sgg \vdash U_1 : U_2}

  \inferrule[\defrule{conv}]
    {\sgg \vdash e:t \and \sgg \vdash u:U \and t \leq u}
    {\sgg \vdash e:u}

  \inferrule[\defrule{pi}]
    {(U_1, U_2, U_3) \in \Rules \\\\
      \sgg \vdash t : U_1 \and
      \sgg (x:t) \vdash u : U_2}
    {\sgg \vdash \prod{x}{t}{u} : U_3}

  \inferrule[\defrule{lam}]
    {\sgg \vdash t : U \and \sgg (x:t) \vdash e:u}
    {\sgg \vdash \abs{x}{\erase{t}}{e} : \prod{x}{t}{u}}

  \inferrule[\defrule{app}]
    {\sgg \vdash e_1 : \prod{x}{t}{u} \and \sgg \vdash e_2 : t}
    {\sgg \vdash \app{e_1}{e_2} : u[x \coloneqq e_2]}

  \inferrule[\defrule{let}]
    {\sgg \vdash e_1:t \and \sgg (x:t \coloneqq e_1) \vdash e_2:u}
    {\sgg \vdash \letin{x}{\erase{t}}{e_1}{e_2} : u[x \coloneqq e_1]}
  \\
  \inferrule[\defrule{ind}]
    {\WF{\sgg}}
    {\sgg \vdash I^s : \indtype{\Sigma}{I}}

  \inferrule[\defrule{constr}]
    {\WF{\sgg}}
    {\sgg \vdash c : \constrtype{\Sigma}{c}{s}}

  \inferrule[\defrule{case}]
    {\sgg \vdash e : \app{I_k^{\hat{s}}}{\overline{p}}{\overline{a}} \and
      \indtype{\Sigma}{I_k} = \prodctx{\any}{\prodctx{\any}{U_k}} \\
      (U_k, U, I_k) \in \Elims \and
      \sgg \vdash P : \motivetype{\Sigma}{\overline{p}}{U}{I_k^{\hat{s}}} \\
      \text{For each $j$:} \and
      \sgg \vdash e_j : \branchtype{\Sigma}{\overline{p}}{c_j}{s}{P}}
    {\sgg \vdash \caseof{\erase{P}}{e}{c_j}{e_j} : \app{P}{\overline{a}}{e}}

  \inferrule[\defrule{fix}]
    {\text{For each $k$:} \and
      t_k \approx \prodctx{\Delta_{k}}{\prod{x_k}{\app{I_k^{\upsilon_k}}{\overline{a}_k}}{u_k}} \and
      \norm{\Delta_{k}} = n_k - 1 \and
      \upsilon_k \pos u_k \\
      \upsilon_k \notin \SV{\Gamma, \overline{a}_k, e_k} \and
      \sgg \vdash t_k : U_k \and
      \sgg \overline{(f_k : t_k)} \vdash e_k : t_k[\upsilon_k \coloneqq \hat{\upsilon}_k]}
    {\sgg \vdash \fix{\seq{n_k}, m}{f_k}{\erase{t_k}^{\upsilon_k}}{e_k} : t_m[\upsilon_m \coloneqq s]}

  \inferrule[\defrule{cofix}]
    {\text{For each $k$:} \and
      t_k \approx \prodctx{\Delta_k}{\app{I_k^{\upsilon_k}}{\overline{a}_k}} \and
      \upsilon_k \neg \Delta_k \\
      \upsilon_k \notin \SV{\Gamma, \overline{a}_k, e_k} \and
      \sgg \vdash t_k : U_k \and
      \sgg \overline{(f_k : t_k)} \vdash e_k : t_k[\upsilon_k \coloneqq \hat{\upsilon}_k]}
    {\sgg \vdash \cofix{m}{f_k}{\erase{t_k}^{\upsilon_k}}{e_k} : t_m[\upsilon_m \coloneqq s]}
\end{mathpar}

\caption{Typing rules}
\label{fig:typing}
\end{figure*}

%%% Local Variables:
%%% TeX-master: "../main.tex"
%%% TeX-engine: default
%%% End:


The typing rules for sized terms are given in \autoref{fig:typing}. As in CIC, we define the three sets \Axioms, \Rules, and \Elims in \autoref{fig:axruel}, which describe how universes are typed, how products are typed, and what eliminations are allowed in case expressions, respectively.
Metafunctions that construct important function types for inductive types, constructors, and case expressions are listed in \autoref{fig:metafunctions}; they are also used by the inference algorithm in \autoref{sec:algorithm}.

Rules \refnorule{var-assum}, \refnorule{const-assum}, \refnorule{univ}, \refnorule{conv} \refnorule{pi}, and \refnorule{app} are essentially unchanged from CIC.
Rules \refnorule{lam} and \refnorule{let} differ only in that type annotations need to be bare to preserve subject reduction,
and the erasure metafunction $\new{\erase{\ph}}$ removes all size annotations from a sized term.

The first significant usage of size annotations are in Rules \refnorule{var-def} and \refnorule{const-def}.
If a variable or a constant is bound to a term in the local or global environment, it is annotated with a size substitution such that the term is well-typed after performing the substitution, allowing for proper $\delta$-/$\Delta$-reduction of variables and constants.
Notably, each usage of a variable or a constant does not have to have the same size annotations.

Inductive types and constructors are typed mostly similar to CIC,
with their types specified by \indtype and \constrtype.
In \refrule{ind}, the \coinductive type itself holds a single size annotation.
In \refrule{constr}, size annotations appear in two places:
\begin{itemize}
    \item In the argument types of the constructor.
      We annotate each occurrence of $I_j$ in the arguments $\Delta^\infty_j$ with a size expression $s$.
    \item On the \coinductive type of the fully-applied constructor,
      which is annotated with the size expression $\hat{s}$.
      Using the successor guarantees that the constructor always constructs an object that is \textit{larger} than any of its arguments of the same type.
\end{itemize}
As an example, consider a possible typing of \text{VCons}:
\begin{align*}
\mt, \mt \vdash \VCons &: \underbrace{(A: \Type{})}_{\text{parameter}} \to \underbrace{(n:\Nat^\infty) \to A \to \Vector^s ~ A ~ n}_{\text{arguments}} \to \underbrace{\Vector^{\hat{s}} ~ A ~ (\Succ ~ n)}_{\text{return type}}
\end{align*}
It has a single parameter $A$ and $\Succ ~ n$ corresponds to the index $\vec{t^\infty}_j$ of the constructor's inductive type.
The input $\Vector$ has size $s$, while the output $\Vector$ has size $\hat{s}$.

In \refrule{case}, a case expression has three parts:
\begin{itemize}
    \item The \textbf{target} $e$ that is being destructed.
      It must have a \coinductive type $I$ with a successor size annotation $\hat{s}_k$ so that recursive constructor arguments have the predecessor size annotation.

    \item The \textbf{motive} $P$, which yields the return type of the case expression.
      While it ranges over the \coinductive's indices,
      the parameter variables $\dom{\Delta_p}$ in the indices' types are bound to the parameters of the target type.
      As usual, the universe of the motive $U$ is restricted by the universe of the \coinductive $U'$ according to \Elims.

      (This presentation follows that of Coq, but differs from that by~\citet{cic-hat-minus, cic-hat-l, cc-hat-omega}, where the case expression contains a return type in which the index and target variables are free and explicitly stated, in the syntactic form $\vec{y}.x.P$.)

    \item The \textbf{branches} $e_j$, one for each constructor $c_j$.
      Again, the parameters of its type are fixed to $\vec{p}$, while ranging over the constructor arguments.
      Note that like in the type of constructors, we annotate each occurrence of $c_j$'s \coinductive type $I$ in $\Delta_j$ with the size expression $s$.
\end{itemize}

Finally, we have the typing of mutual \cofixpoints in rules \refnorule{fix} and \refnorule{cofix}.
We take the annotated type $t_k$ of the $k$th \cofixpoint definition to be convertible to a function type containing a \coinductive type, as usual.
However, instead of the guard condition, we ensure termination/productivity using size expressions.

The main difficulty in these rules is supporting size preserving \cofixpoints.
We must restrict how the size variable $v_k$ appears in the type of the \cofixpoints using the positivity and negativity judgments.
For fixpoints, the type of the $n_k$th argument, the recursive argument, is an inductive type annotated with a size variable $v_k$.
For cofixpoints, the return type is a coinductive type annotated with $v_k$.
The positivity or negativity of $v_k$ in the rest of $t_k$ indicate where $v_k$ may occur other than in the \corecursive position.
For instance, supposing that $n = 1$,
$\app{\List^\upsilon}{\Nat} \to \app{\List}{\Nat} \to \app{\List^\upsilon}{\Nat}$
is a valid fixpoint type with respect to $\upsilon$, while
$\app{\List^\upsilon}{\Nat} \to \app{\List^\upsilon}{\Nat} \to \app{\Stream^\upsilon}{\Nat}$
is not, since $\upsilon$ illegally appears negatively in $\Stream$ and must not appear at all in the parameter of the second $\List$ argument type.
This restriction ensures the aforementioned monotonicity property of subtyping for the \cofixpoints' types,
so that $u_k \leq \subst{u_k}{\upsilon_k}{\succ{\upsilon}_k}$ holds for fixpoints,
and that $\subst{u}{\upsilon_k}{\succ{\upsilon}_k} \leq u$ for each type $u$ in $\Delta_k$ holds for cofixpoints.

As in \refrule{lam}, to maintain subject reduction, we cannot keep the size annotations, instead replacing them with position variables.
The metafunction $\new{\erase{\ph}^\upsilon}$ replaces $\upsilon$ annotations with the position annotation $\ast$ and erases all other size annotations.

Checking termination and productivity is then relatively straightforward.
If $t_k$ are well typed, then the \cofixpoint bodies should have type $t_k$ with a successor size when $(\assm{f_k}{t_k})$ are in the local environment.
This tells us that the recursive calls to $f_k$ in fixpoint bodies are on smaller-sized arguments, and that corecursive bodies produce objects with size larger than those from the corecursive call to $f_k$.
The type of the $m$th \cofixpoint is then the $m$th type $t_m$ with some size expression $s$ substituted for the size variable $v_m$.

In Coq, the indices of the recursive elements are often elided, and there are no user-provided position annotations at all.
We show how indices and position annotations can be computed during size inference in \autoref{sec:algorithm}.

%%% Local Variables:
%%% TeX-master: "../main.tex"
%%% TeX-engine: default
%%% End:
