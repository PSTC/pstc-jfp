\section{Simplified \titlelang}
We present a simplified variant of \lang without mutually-defined \coinductive types and mutual \cofixpoints.
While adhering less to Gallina, this is closer to \CIChatminus and \CIChat, and is easier to reason about.
Going from the proper \lang to the simplified \lang is simple, and we provide the simplified variant explicitly for convenience.
The modified portions of the syntax of terms and \coinductive definitions for simplified \lang are given in \autoref{fig:simpl-terms} and \autoref{fig:simpl-inductives}.

\begin{figure}
\centering
\begin{align*}
T[\alpha] &\Coloneqq \dots \\
    &\mid \fix*{n}{\mathcal{X}}{T^*}{T[\alpha]} \\
    &\mid \cofix*{}{\mathcal{X}}{T^*}{T[\alpha]}
\end{align*}
\caption{Syntax of \lang terms with no mutual \cofixpoints (excerpt)}
\label{fig:simpl-terms}
\end{figure}

\begin{figure}
\centering
\begin{align*}
\textit{Ind} &\Coloneqq \mathcal{I}: \Pi\Delta . \Pi\Delta^\infty . U \coloneqq \langle \mathcal{C}: \Pi\Delta^\infty . \mathcal{I} ~ \overline{\mathcal{X}} ~ \overline{T^\infty} \rangle
\end{align*}
\begin{equation*}
I: \Pi\Delta_p. \Pi\Delta_i . \varw_I \coloneqq \langle c_j: \Pi\Delta_j . I ~ \dom{\Delta_p} ~ \overline{t}_j \rangle
\end{equation*}
\caption{(Co)inductive definitions with no mutually-defined inductive types}
\label{fig:simpl-inductives}
\end{figure}


We give the corresponding typing rules for case expression and \cofixpoints in \autoref{fig:simpl-typing}, as well as the required metafunctions in \autoref{fig:simpl-metafunctions}. The main differences to note are that only a single size expression is passed to the metafunctions, and we no longer need indices to differentiate between different mutually-defined \coinductive types in the case rule or to differentiate between different mutual \cofixpoint definitions in the (co)fix rules.

\begin{figure}
\centering
\begin{align*}
    \indtype{\Sigma}{I} &=
        \Pi \Delta_p . \Pi \Delta_i . \varw_I \\
    \constrtype{\Sigma}{c_\ell}{s} &=
        \Pi \Delta_p . \Pi \Delta_\ell [I^\infty \coloneqq I^s] . I^{\hat{s}} ~ \dom{\Delta_p} ~ \overline{t}_\ell \\
    \motivetype{\Sigma}{\overline{p}}{\varw}{I^s} &=
        \Pi \Delta_i[\dom{\Delta_p} \coloneqq \overline{p}] . \Pi \_ : I^s ~ \overline{p} ~ \dom{\Delta_i} . \varw \\
    \branchtype{\Sigma}{\overline{p}}{c_\ell}{s}{\wp} &=
        \Pi \Delta_\ell [I^\infty \coloneqq I^s][\dom{\Delta_p} \coloneqq \overline{p}] . \wp ~ \overline{t}_\ell ~ (c_\ell ~ \overline{p} ~ \dom{\Delta_\ell})
\end{align*}
\begin{displaymath}
    \textit{where}\;\:
    \ell \in \overline{\jmath},
    \big(I: \Pi\Delta_p . \Pi\Delta_i . \varw_I \coloneqq \langle c_j : \Pi \Delta_j . I_j ~ \_ ~ \overline{t}_j \rangle \big) \in \Sigma
\end{displaymath}
\caption{Metafunctions for typing rules}
\label{fig:simpl-metafunctions}
\end{figure}

\begin{figure}
\centering
\begin{mathpar}
    \inferrule*[right=(case)]{
        \sgg \vdash e : I^{\hat{s}} ~ \overline{p} ~ \overline{a} \\
        \indtype{\Sigma}{I} = \Pi \_ . \Pi \_ . \varw_I \qquad (\varw_I, \varw, I) \in \Elims \\\\
        \sgg \vdash \wp : \motivetype{\Sigma}{\overline{p}}{\varw}{I^{\hat{s}}} \\
        \sgg \vdash e_j : \branchtype{\Sigma}{\overline{p}}{c_j}{s}{\wp}
    }{
        \sgg \vdash \caseof{|\wp|}{e}{c_j}{e_j} : \wp\overline{a}e
    }
    \hva \and
    \inferrule*[right=(fix)]{
        t \approx \Pi \Delta_1 . \Pi x : I^\upsilon ~ \overline{a}. \Pi \Delta_2 . u \\
        \|\Delta_1\| = n - 1 \\
        \upsilon \pos \Delta_1, \Delta_2, u \\
        \upsilon \notin \SV{\Gamma, \overline{a}, e} \\
        \upsilon, \lfloor s \rfloor \in \mathcal{V}^* \\
        \sgg \vdash t : \varw \\
        \sgg (f : t) \vdash e : t[\upsilon \coloneqq \hat{\upsilon}]
    }{
        \sgg \vdash \fix*{n}{f}{|t|^*}{e} : t[\upsilon \coloneqq s]
    }
    \hva \and
    \inferrule*[right=(cofix)]{
        t \approx \Pi \Delta . I^\upsilon ~ \overline{a} \\
        \upsilon \neg \Delta \\
        \upsilon \notin \SV{\Gamma, \overline{a}, e} \\
        \upsilon, \lfloor s \rfloor \in \mathcal{V}^* \\
        \sgg \vdash t : \varw \\
        \sgg (f : t) \vdash e : t[\upsilon \coloneqq \hat{\upsilon}]
    }{
        \sgg \vdash \cofix*{}{f}{|t|^*}{e_k} : t[\upsilon \coloneqq s]
    }
\end{mathpar}
\caption{Typing rules (excerpt)}
\label{fig:simpl-typing}
\end{figure}


Finally, we give the corresponding size inference algorithm for these terms in \autoref{fig:simpl-algorithm}. Again, we do away with some of the indices, and \RecCheckLoop now takes a single size expression, a single term, and a single type.
The modified version of \RecCheckLoop is provided in \autoref{fig:simpl-helpers}.

\begin{figure}
\centering
\begin{mathpar}
\inferrule*[right=(a-case)]{
    \cgg \vdash e^\circ \rightsquigarrow C_1, e \Rightarrow^* I^s ~ \overline{p} ~ \overline{a} \\
    C_1, \Gamma_G, \Gamma \vdash \wp^\circ \rightsquigarrow C_2, \wp \Rightarrow t_p \\
    \prodctx{\_}{\prodctx{\Delta_i}{\varw_I}} = \indtype{\Sigma}{I} \\
    (\_, \varw) = \decompose{t_p}{\|\Delta_i\| + 1} \\
    \elim{\varw_I}{\varw}{I} \\
    \upsilon = \fresh{1} \\
    C_3 = \casesize{I^s}{\hat{\upsilon}} \\
    C_4 = t_p \preceq \motivetype{\Sigma}{\overline{p}}{\varw}{I^{\hat{\upsilon}}} \\
    C_5 = C_2 \cup C_3 \cup C_4 \\
    C_5, \Gamma_G, \Gamma \vdash e^\circ_j \Leftarrow \branchtype{\Sigma}{\overline{p}}{c_j}{\upsilon}{\wp} \rightsquigarrow C_{6j}, e_j \\
    C_6 = \textstyle\bigcup_j C_{6j}
}{
    \cgg \vdash \caseof{\wp^\circ}{e^\circ}{c_j}{e_j^\circ} \rightsquigarrow C_6, \caseof{|\wp|}{e}{c_j}{e_j} \Rightarrow \wp\overline{a}e
}
\hva \and
\inferrule*[right=(a-fix)]{
    \cgg \vdash t^\circ \rightsquigarrow \_, \_ \Rightarrow \_ \\\\
    \cgg \vdash \setrecstars{t^\circ}{n} \rightsquigarrow C_1, t \Rightarrow^* \varw \\\\
    C_1, \Gamma_G, \Gamma(f : t) \vdash e^\circ \Leftarrow  \shift{t} \rightsquigarrow C_2, e \\
    C_3 = \RecCheckLoop{C_2}{\Gamma}{\getrecvar{t}{n}}{t}{e}
}{
    \cgg \vdash \fix*{n}{f_k}{t^\circ}{e} \rightsquigarrow C_3, \fix*{n}{f}{|t|^*}{e} \Rightarrow t
}
\hva \and
\inferrule*[right=(a-fix)]{
    \cgg \vdash t^\circ \rightsquigarrow \_, \_ \Rightarrow \_ \\\\
    \cgg \vdash \setcorecstars{t^\circ} \rightsquigarrow C_1, t \Rightarrow^* \varw \\\\
    C_1, \Gamma_G, \Gamma(f : t) \vdash e^\circ \Leftarrow  \shift{t} \rightsquigarrow C_2, e \\
    C_3 = \RecCheckLoop{C_2}{\Gamma}{\getcorecvar{t}}{t}{e}
}{
    \cgg \vdash \cofix*{}{f_k}{t^\circ}{e} \rightsquigarrow C_3, \cofix*{}{f}{|t|^*}{e} \Rightarrow t
}
\end{mathpar}
\caption{Size inference algorithm (excerpt)}
\label{fig:simpl-algorithm}
\end{figure}

%%% Local Variables:
%%% TeX-master: "../main.tex"
%%% TeX-engine: default
%%% End:

\begin{figure}
\centering

\begin{minted}[escapeinside=<>,mathescape=true]{ocaml}
                let rec RecCheckLoop <$C_2$> <$\tau$> <$t$> <$e$> =
                  try let pv = <$\texttt{PV}$> <$t$> in
                      let sv = (<$\texttt{SV}$> <$t$> <$\cup$> <$\texttt{SV}$> <$e$>) <$\setminus$> pv in
                      RecCheck <$C_2$> <$\tau$> pv sv
                  with RecCheckFail <$V$> ->
                      <$\mathcal{V}^*$> := <$\mathcal{V}^* \setminus V$>;
                      RecCheckLoop <$C_2$> <$\tau$> <$t$> <$e$>
\end{minted}

\caption{Modified pseudocode implementation of \textsc{RecCheckLoop}}
\label{fig:simpl-helpers}
\end{figure}

%%% Local Variables:
%%% TeX-master: "../main.tex"
%%% TeX-engine: default
%%% End:

