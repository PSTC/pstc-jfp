\section{Reduction, Convertibility, Takahashi Translation}\label{sec:red-conv-trans}

\begin{figure}
\begin{flushleft}
\fbox{$\gg \vdash T \reduce_{\beta\zeta\delta\Delta\iota\mu\nu} T$}
\end{flushleft}

\vspace{-3ex}

\begin{mathpar}
  \inferrule*[right=\defnamerule{beta}{$\beta$}]{~}
    {\gg \vdash \app{(\abs{x}{t^\circ}{e_1})}{e_2} \reduce_\beta \subst{e_1}{x}{e_2}}
  \and
  \inferrule*[right=\defnamerule{zeta}{$\zeta$}]{~}
    {\gg \vdash \letin{x}{t^\circ}{e_1}{e_2} \reduce_\zeta \subst{e_2}{x}{e_1}}
  \and
  \inferrule[\defnamerule{delta-local}{$\delta$-local}]
    {(x : t \coloneqq e) \in \Gamma \and \SV(e, t) = \vec{\upsilon_i}}
    {\gg \vdash x^{\seq{s_i}} \reduce_\delta e \overline{[\upsilon_i \coloneqq s_i]}}
  \and
  \inferrule[\defnamerule{Delta-global}{$\Delta$-global}]
    {(\Defn{x}{t}{e}) \in \Gamma_G \and \SV(e, t) = \vec{\upsilon_i}}
    {\gg \vdash x^{\seq{s_i}} \reduce_\Delta e \overline{[\upsilon_i \coloneqq s_i]}}
  \and
  \inferrule*[right=\defnamerule{iota}{$\iota$}]{~}
    {\gg \vdash \caseof{P^\circ}{(\app{c_\ell}{\vec{p}}{\vec{a}})}{c_j}{e_j} \reduce_\iota \app{e_\ell}{\vec{a}}}
  \and
  \inferrule*[right=\defnamerule{mu}{$\mu$}]
    {q_i = \fix{\seq{n_k}, i}{f_k}{t_k}{e_k} \and \norm{\vec{b}} = n_m - 1}
    {\gg \vdash \app{q_m}{\vec{b}}{(\app{c_\ell}{\vec{p}}{\vec{a}})}
      \reduce_\mu \app{\substvec{e_m}{f_k}{q_k}}{\vec{b}}{(\app{c_\ell}{\vec{p}}{\vec{a}})}}
  \and
  \inferrule*[right=\defnamerule{nu}{$\nu$}]
    {q_i = \cofix{i}{f_k}{t_k}{e_k}}
    {\gg \vdash \caseof{P^\circ}{(\app{q_m}{\vec{b}})}{c_j}{a_j}
      \reduce_\nu \caseof{P^\circ}{(\app{\substvec{e_m}{f_k}{q_k}}{\vec{b}})}{c_j}{a_j}}
\end{mathpar}
\caption{Reduction rules}
\label{fig:reduction-full}
\end{figure}

\begin{figure}
\begin{flushleft}
\fbox{$\gg \vdash T \reduce_{\beta\zeta\delta\Delta\iota\mu\nu'} T$}
\end{flushleft}

\vspace{-6ex}

\begin{mathpar}
  \vdots
  \\
  \inferrule*[right=\defnamerule{nupr}{$\nu'$}]
    {q_i = \cofix{i}{f_k}{t_k}{e_k}}
    {\gg \vdash q_m
      \reduce_{\nu'} \substvec{e_m}{f_k}{q_k}}
\end{mathpar}
\caption{Reduction rules (with unrestricted cofixpoint reduction)}
\label{fig:reduction-full-unrestricted}
\end{figure}

\begin{figure}
\begin{flushleft}
\fbox{$\gg \vdash T \conv T$}
\end{flushleft}

\vspace{-3ex}

\begin{mathpar}
  \inferrule[\defnamerule{conv-red}{conv-$\reduce^*$}]
    {\gg \vdash e_1 \reduce^* e \\\\
      \gg \vdash e_2 \reduce^* e}
    {\gg \vdash e_1 \conv e_2}
  \and
  \inferrule[\defnamerule{conv-eta-r}{conv-$\eta$-l}]
    {\gg \vdash e_1 \reduce^* \abs{x}{\erase{t}}{e} \\\\
      \gg \vdash e_2 \reduce^* e'_2 \\\\
      \gg (x:t) \vdash e \conv e'_2 ~ x}
    {\gg \vdash e_1 \conv e_2}
  \and
  \inferrule[\defnamerule{conv-eta-1}{conv-$\eta$-r}]
    {\gg \vdash e_1 \reduce^* e'_1 \\\\
      \gg \vdash e_2 \reduce^* \abs{x}{\erase{t}}{e} \\\\
      \gg (x:t) \vdash e'_1 ~ x \conv e}
    {\gg \vdash e_1 \conv e_2}
\end{mathpar}
\caption{Convertibility rules}
\label{fig:conversion}
\end{figure}


\autoref{fig:reduction-full} lists the complete definitions for all reduction rules, including our new rules for $\delta$- and $\Delta$-reduction.
Notice that in fixpoints, the $n_m$th recursive argument needs to be an applied constructor, while cofixpoints can only be reduced as a case expression target.
These conditions are required for strong normalization.
However, this restricted cofixpoint unfolding causes issues with subject reduction~\citep{cc-hat-omega}.
\autoref{fig:reduction-full-unrestricted} presents an alternate unrestricted nonterminating $\nu'$-reduction of cofixpoints outside of case expressions so that subject reduction holds at the cost of normalization.

We define $\reduce$ as the least compatible closure of $\reduce_{\beta\zeta\delta\Delta\iota\mu\nu}$, $\reduce_n$ as the $n$-step reduction of $\reduce$, and $\reduce^*$ as the reflexive--transitive closure of $\reduce$.
Convertibility ($\approx$) is the symmetric closure of $\reduce^*$ up to $\eta$-expansion, and is formally defined in \autoref{fig:conversion}.

The Takahashi translation $e^\dagger$ of a term $e$~\citep{takahashitrans} is used in our proof of confluence.
Informally, we define it as the simultaneous single-step reduction of all $\beta\zeta\delta\Delta\iota\mu\nu$-redexes of $e$ in left-most inner-most order.
We further define $e^{n\dagger}$ as the $n$-step Takahashi translation of $e$.

%%% Local Variables:
%%% TeX-master: "../main.tex"
%%% TeX-engine: default
%%% End:
