\section{Introduction}\label{sec:intro}

Proof assistants based on dependent type theory rely on the termination of recursive functions and the productivity of corecursive functions to ensure two important properties: logical consistency, so that it is not possible to prove false propositions; and decidability of type checking, so that checking that a program proves a given proposition is decidable.

In proof assistants such as Coq, termination and productivity are enforced by a \emph{guard predicate} on fixpoints and cofixpoints respectively.
For fixpoints, recursive calls must be \emph{guarded by destructors}; that is, they must be performed on structurally smaller arguments.
For cofixpoints, corecursive calls must be \emph{guarded by constructors}; that is, they must be the structural arguments of a constructor.
The following examples illustrate these structural conditions.

\begin{minted}{coq}
Fixpoint plus n m : nat :=
  match n with
  | O => m
  | S p => S (plus p m)
  end.
CoFixpoint const {A} a : Stream A := Cons a (const a).
\end{minted}

In the recursive call to \coqinline{plus}, the first argument \coqinline{p} is structurally smaller than \coqinline{S p}, which is the shape of the original first argument \coqinline{n}. Similarly, in \coqinline{const}, the constructor \coqinline{Cons} is applied to the corecursive call.

The actual implementation of the guard predicate extends beyond the \guardedbydestructors and \guardedbyconstructors conditions
to accept a larger set of terminating and productive functions.
In particular, function calls will be unfolded (\ie inlined) in the bodies of \cofixpoints
and reduction will be performed if needed before checking the guard predicate.
This has a few disadvantages:
firstly, the bodies of these functions are required, which hinders modular design;
and secondly, as aptly summarized by \citet{coqterm},

\begin{quote}
  ... unfold[ing] all the definitions used in the body of the function, do[ing] reductions, \etc
  ... makes typechecking extremely slow at times.
  Also, the unfoldings can cause the code to bloat by orders of magnitude and become impossible to debug.
\end{quote}

Furthermore, changes in the structural form of functions used in \cofixpoints can cause the guard predicate to reject the program even if the functions still behave the same.
The following simple example, while artificial, illustrates this structural fragility.
\begin{minted}{coq}
Fixpoint minus n m : nat :=
  match n, m with
  | O, _ => n
  | _, O => n
  | S n', S m' => minus n' m'
  end.
\end{minted}
\begin{minted}{coq}
Fixpoint div n m : nat :=
  match n with
  | O => O
  | S n' => S (div (minus n' m) m)
  end.
\end{minted}

If we replace \coqinline{| O, _ => n} with \coqinline{| O, _ => O} in \coqinline{minus}, the behaviour doesn't change, but \coqinline{O} is not a structurally smaller term of \coqinline{n} in the recursive call to \coqinline{div}, so \coqinline{div} no longer satisfies the guard predicate.
The acceptance of \coqinline{div} then depends on a separate definition independent of \coqinline{div}.
While the difference is easy to spot here, for larger programs or programs that use many imported library definitions,
this behaviour can make debugging much more difficult.
Furthermore, the guard predicate is unaware of the obvious fact that \coqinline{minus} never returns a \coqinline{nat} larger than its first argument, which the user would have to prove in order for \coqinline{div} to be accepted with our alternate definition of \coqinline{minus}.

In short, the extended syntactic guard condition long used by Coq is anti-modular, anti-compositional, has poor performance characteristics, and requires the programmer to either avoid certain algorithms or pay a large cost in proof burden.

\paragraph*{} This situation is particularly unfortunate, as there exists a non-syntactic termination- and productivity-checking method that overcomes these issues,
whose theory is nearly as old as the guard condition itself: sized types.

In essence, the \coinductive type of a construction is annotated with a size, which provides some information about the size of the construction.
In this paper, we consider a simple size algebra: \mbox{$s \coloneqq \upsilon \mid \hat{s} \mid \infty$}, where $\upsilon$ ranges over size variables.
If the argument to a constructor has size $s$, then the fully-applied constructor would have a successor size $\hat{s}$.
For instance, the constructors for the naturals follow the below rules:

\vspace{-2ex}
\begin{mathpar}
  \inferrule*
    {~}
    {\Gamma \vdash \Zero : \Nat^{\hat{s}}}
  \and
  \inferrule*
    {\Gamma \vdash n : \Nat^s}
    {\Gamma \vdash \app{\Succ}{n} : \Nat^{\hat{s}}}
\end{mathpar}

Termination and productivity checking is then \emph{just} a type checking rule that uses size information.
For termination, the recursive call must be done on a construction with a smaller size, so when typing the body of the fixpoint, the reference to itself in the typing context must have a smaller size.
For productivity, the returned construction must have a larger size than that of the corecursive call, so the type of the body of the cofixpoint must be larger than the type of the reference to itself in the typing context.
In short, they both follow the following (simplified) typing rule, where $\upsilon$ is an arbitrary fresh size variable annotated on the \coinductive types, and $s$ is an arbitrary size expression as needed.

\vspace{-2ex}
\begin{mathpar}
  \inferrule*[]
    {\Gamma (\assm{f}{t^\upsilon}) \vdash e: t^{\hat{\upsilon}}}
    {\Gamma \vdash \kw{(co)fix}\ f \mathbin{:} t \mathbin{\coloneqq} e : t^s}
\end{mathpar}

We can then assign \coqinline{minus} the type $\text{Nat}^\upsilon \to \text{Nat} \to \text{Nat}^\upsilon$.
The fact that we can assign it a type indicates that it will terminate,
and the $\upsilon$ annotations indicate that the function preserves the size of its first argument.
Then \coqinline{div} uses only the type of \coqinline{minus} to successfully type check, not requiring its body.
Furthermore, being type-based and not syntax-based, replacing \coqinline{| O, _ => n} with \coqinline{| O, _ => O}
doesn't affect the type of \coqinline{minus} or the typeability of \coqinline{div}.
Similarly, some other \cofixpoints that preserve the size of arguments in ways that aren't syntactically obvious may be typed to be size preserving,
expanding the set of terminating and productive functions that can be accepted.
Finally, if additional expressivity is needed, rather than using syntactic hacks like inlining, we could take the semantic approach of enriching the size algebra.

\paragraph*{} It seems perfect; so why doesn't Coq \emph{just} use sized types?

\noindent That is the question we seek to answer in this paper.

Unfortunately, past work on sized types~\citep{cic-hat, cic-hat-minus,cc-hat-omega} for the Calculus of (Co)\-Inductive Constructions (CIC), Coq's underlying calculus, have some practical issues:

\begin{itemize}
    \item They require nontrivial backward-incompatible additions to the surface language,
      such as size annotations on \cofixpoint types and polarity annotations on \coinductive definitions.
    \item They are missing important features found in Coq such as global and local definitions, and universe cumulativity.
    \item They restrict size variables from appearing in terms, which precludes, for instance,
      defining type aliases for sized types.
\end{itemize}

To resolve these issues, we extend \CIChat~\citep{cic-hat}, \CIChatminus~\citep{cic-hat-minus-nat,cic-hat-minus}, and \CChatomega~\citep{cc-hat-omega} in our calculus \textbf{\lang} (``\emph{CIC-star-hat}''),
and design a size inference algorithm from CIC to \lang,
borrowing from the algorithms in past work~\citep{f-hat, cic-hat, cc-hat-omega}.
\autoref{table:CICs} summarizes the differences between \lang and these past
works; we give a detailed comparison in \autoref{sec:ancestry}.

\newcommand{\cmark}{\ding{51}}
\newcommand{\xmark}{\ding{55}}
\begin{table}
\centering
\begin{tabular}{| l | c | c | c | c |}
\hline
\textbf{Feature} & \textit{\CIChat} & \textit{\CIChatminus} & \textit{Coq} & \textit{\lang} \\
\hline
Universe cumulativity & \xmark & \xmark & \cmark & \cmark \\
Definitions           & \xmark & \xmark & \cmark & \cmark \\
Parameter polarities  & \cmark & \cmark & \xmark & \xmark \\
Nested \coinductives  & \cmark & \cmark & \cmark & \xmark \\
Normalization proven? & \xmark & \cmark & ?      & \xmark \\
Size inference algorithm & \cmark & \xmark & N/A & \cmark \\
\hline
\end{tabular}
\caption{Comparison of the features in \CIChat, \CIChatminus, Coq, and \lang}
\label{table:CICs}
\end{table}

For \lang we prove confluence and subject reduction.
However, new difficulties arise when attempting to prove strong normalization and consistency.
Proof techniques from past work, especially from \citet{cic-hat-minus}, don't readily adapt to our modifications,
in particular to universe cumulativity and unrestricted size variables.
On the other hand, set-theoretic semantics of type theories that do have these features don't readily adapt to the interpretation of sizes, either,
with additional difficulties due to untyped conversion.
We detail a proof attempt on a variant of \lang and discuss its shortcomings.

Even supposing that the metatheoretical problems can be solved and strong normalization and consistency proven,
is an implementation of this system practical?
Seeking to answer this question, we have forked Coq~\citep{impl}, implemented the size inference algorithm within its kernel,
and opened a draft pull request to the Coq repository%
\footnote{\url{https://github.com/coq/coq/pull/12426/} (now closed).}.
To maximize backward compatibility, the surface language is completely unchanged,
and sized typing can be enabled by a flag that is off by default.
This flag can be used in conjunction with or without the existing guard checking flag enabled.

While sized typing enables many of our goals, namely increased expressivity with modular and compositional typing for \cofixpoints, the performance cost is unacceptable.
We measure at least a $5.5 \times$ increase in compilation time in some standard libraries.
Much of the performance cost is intrinsic to the size inference algorithm, and thus intrinsic to attempting to maintain backward compatibility.
We analyze the performance of our size inference algorithm and our implementation in detail.

So why doesn't Coq \emph{just} use sized types?
Because it seems it must either sacrifice backward compatibility or compile-time performance,
and the lack of a proof of consistency may be a threat to Coq's trusted core.
While nothing yet leads us to believe that \lang is inconsistent,
the performance sacrifice required for compatibility makes our approach seem wildly impractical.

\paragraph*{} The remainder of this paper is organized as follows.
We formalize \lang in \autoref{sec:typing},
and discuss the desired metatheoretical properties in \autoref{sec:metatheory}.
In \autoref{sec:algorithm}, we present the size inference algorithm from unsized terms to sized \lang terms,
and evaluate an implementation in our fork in \autoref{sec:impl}.
While prior sections all handle the formalization metatheory of \lang,
\autoref{sec:impl} contains the main analysis and results on the performance.
Finally, we take a look at all of the past work on sized types leading up to \lang in \autoref{sec:related}, and conclude in \autoref{sec:conclusion}.

%%% Local Variables:
%%% TeX-master: "../main.tex"
%%% TeX-engine: default
%%% End:
