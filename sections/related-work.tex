\section{Related Work}\label{sec:related}

The history of sized types is vast and varied.
Extensive prior accounts are given in dissertations by \citet{lambda-hat-diss} and \citet{abel-diss}.
Here, we focus on two lineages towards sized dependent type theories: first, the more-or-less direct ancestry of \lang, and second, a contrasting line of work on type systems with explicit higher-order sized types.

\subsection{Ancestry of \titlelang}

Perhaps the most well-known work on sized types is by \citet{hughes},
who introduce sized types for a Hindley--Milner type system with \coinductives and a size inference algorithm,
as well as the term ``sized types''.
Their size algebra is more expressive than ours, with size addition $s_1 + s_2$ and constant multiplication $n \times s$.
Independently, \citet{ccr} introduces \CCR, a Calculus of Constructions (CoC) with \textit{guarded types},
a type-based termination checking alternative to the earlier syntactic guard condition~\citep{guard-condition}.
There, types are guarded with a type operator $\widehat{\ph}$,
similar to the later modality $\rhd \ph$ in modern guarded type theories.
Based on a semantic interpretation of \CCR,
\citet{acg} introduce a simply-typed lambda calculus (STLC) with coinductives with \textit{type labels},
corresponding roughly to size annotations with successor sizes.

Following this, \citet{lambda-hat} and \citet{lambda-hat-diss} introduce \lambdahat,
another STLC with inductives and size annotations with the same size algebra we use,
although they are instead called \textit{stages}.
It improves upon the work of \citet{acg} by adding an implicit form of size polymorphism:
the position size variable of fixpoint types are substituted by an arbitrary size expression,
just as in \refrule{fix}.
\citet{f-hat} extend \lambdahat to System F with \Fhat, and introduce and prove correct a size inference algorithm.
This includes the \RecCheck algorithm that we use.
They continue on to extend \Fhat with \textit{sized products} (that is, pairs with size annotations) in \FXhat~\citep{fx-hat}, whose size expressions include size addition, and to CIC in \CIChat~\citep{cic-hat}.
Our size inference algorithm is based directly on that of \CIChat.
We add to it global and local definitions and explicitly treat mutually-defined \coinductives and \cofixpoints,
while removing polarities and subtyping based on these polarities.

However, normalization of \CIChat is only a conjecture; it is later proven for the restricted language \CIChatminus by \citet{cic-hat-minus-nat} (with only naturals) and by \citet{cic-hat-minus} (with inductive types).
The restrictions include disallowing size variables in the bodies of functions, in the arguments of applications, in the branches of case expressions, and in the indices of inductives; erasing the parameters to constructors; and disallowing strong elimination to types with size variables.
We remove these restrictions to allow using sized definitions and for backward compatibility with Coq.

Our typing rules and inference algorithm for coinductives and cofixpoints are based on \CChatomega~\citep{cc-hat-omega},
which extends CoC with sized coinductive streams.
Further extensions to the size algebra are linear sized types in \CIChatl~\citep{cic-hat-l},
which adds constant multiplication to a sized CoC with naturals and streams;
and well-founded sized types in \CIChatsub~\citep{wellfounded},
which changes the premise type checking the \cofixpoint body in Rules \refnorule{fix} and \refnorule{cofix}
to the recursive reference having \emph{any} smaller size according to the subsizing rules,
rather than the direct predecessor.
All three include size inference algorithms similar to that of \CIChat.

There are prototype implementations of \CIChatminus~\citep{cicminus} and \CIChatsub~\citep{cic-wf}.
It appears that there were also plans to implement sized types in Coq by \citet{coq-hat}, but seem to be abandoned.

\subsection{Higher-Order Sized Types}

For the purposes of size inferrability from unannotated code,
the type systems from \lambdahat up to \CIChat and its variations treat sizes as merely annotations
and feature only what can be considered as prenex size polymorphism.
On the other hand, \citet{abel-diss} introduces \Fomegahat, a sized type system for System F$_\omega$ that treats size as a kind,
which therefore allows for higher-order size polymorphism via explicit quantification.
While \Fomegahat subsumes \Fhat and uses the same size algebra,
it uses recursive and corecursive type constructors ($\mu$- and $\nu$-types) rather than inductive (and coinductive) type definitions.

Higher-order sized types of the same flavour are implemented in a dependently-typed setting in MiniAgda~\citep{miniagda}.
To avoid inconsistencies introduced by the interplay between sized types and pattern matching,
it also introduces bounded size patterns \mbox{$\upsilon_1 < \upsilon_2$}.
\citet{flationary} expands upon the theoretical side with bounded size quantification \mbox{$\forall \upsilon \mathrel{<} s\mathpunct{.} t$} and well-founded recursion on sizes,
which are also implemented in MiniAgda.
\citet{f-omega-cop} combine well-founded sized types and copatterns for System F$_\omega$ with \corecursive type constructors in \Fomegacop (which was cited as the inspiration for \CIChatsub).

\cite{nbe} prove normalization of a higher-order sized dependent type theory with naturals, but without bounded size quantification.
To our knowledge, this is the only formalization of higher-order sized dependent types in the literature.
Additionally, they also add a notion of \emph{size irrelevance} so that more terms are convertible when their sizes don't matter;
this is an orthogonal feature that can be added to \lang as well,
since the issues size irrelevance aims to resolve can also arise here.

Sized types with higher-order bounded size quantification are implemented in Agda%
\footnote{For Agda's documentation on sized types, see \url{https://agda.readthedocs.io/en/latest/language/sized-types.html}.};
however, it is known to be inconsistent%
\footnote{For a detailed discussion on the issue, see \url{https://github.com/agda/agda/issues/2820}.}.
In short, it is possible to express the well-foundedness of sizes within Agda,
but the infinite size $\infty$ itself is \emph{not} well founded,
as $\infty + 1 = \infty$ and $\infty < \infty$ hold,
making it possible to derive a contradiction.

%%% Local Variables:
%%% TeX-master: "../main.tex"
%%% TeX-engine: default
%%% End:
