\section{Related Work}\label{sec:related}

The history of sized types is vast and varied.
Extensive prior accounts are given in dissertations by \citet{lambda-hat-diss} and \citet{abel-diss}.
Here, we focus on two lineages towards sized dependent type theories: first, the more-or-less direct ancestry of \lang, and second, a contrasting line of work on type systems with explicit higher-order sized types.

\subsection{Ancestry of \titlelang}\label{sec:ancestry}

Perhaps the most well-known work on sized types is by \citet{hughes},
who introduce sized types for a Hindley--Milner type system with \coinductives and a size inference algorithm,
as well as the term ``sized types''.
Their size algebra is more expressive than ours, with size addition $s_1 + s_2$ and constant multiplication $n \times s$.
Independently, \citet{ccr} introduces \CCR, a Calculus of Constructions (CoC) with \textit{guarded types},
a type-based termination checking alternative to the earlier syntactic guard condition~\citep{guard-condition}.
There, types are guarded with a type operator $\widehat{\ph}$,
similar to the later modality $\rhd \ph$ in modern guarded type theories.
Based on a semantic interpretation of \CCR,
\citet{acg} introduce a simply-typed lambda calculus (STLC) with coinductives with \textit{type labels},
corresponding roughly to size annotations with successor sizes.

Following this, \citet{lambda-hat} and \citet{lambda-hat-diss} introduce \lambdahat,
another STLC with inductives and size annotations with the same size algebra we use,
although they are instead called \textit{stages}.
It improves upon the work of \citet{acg} by adding an implicit form of size polymorphism:
the position size variable of fixpoint types are substituted by an arbitrary size expression,
just as in \refrule{fix}.
\citet{f-hat} extend \lambdahat to System F with \Fhat, and introduce and prove correct a size inference algorithm.
This includes the \RecCheck algorithm that we use.
They continue on to extend \Fhat with \textit{sized products} (that is, pairs with size annotations) in \FXhat~\citep{fx-hat}, whose size expressions include size addition, and to CIC in \CIChat~\citep{cic-hat}.
Our size inference algorithm is based directly on that of \CIChat.
We add to it global and local definitions and explicitly treat mutually-defined \coinductives and \cofixpoints,
while removing polarities and subtyping based on these polarities.

However, normalization of \CIChat is only a conjecture; it is later proven for the restricted language \CIChatminus by \citet{cic-hat-minus-nat} (with only naturals) and by \citet{cic-hat-minus} (with inductive types).
The restrictions include disallowing size variables in the bodies of functions, in the arguments of applications, in the branches of case expressions, and in the indices of inductives; erasing the parameters to constructors; and disallowing strong elimination to types with size variables.
We remove these restrictions to allow using sized definitions and for backward compatibility with Coq.

Our typing rules and inference algorithm for coinductives and cofixpoints are based on \CChatomega~\citep{cc-hat-omega},
which extends CoC with sized coinductive streams.
Further extensions to the size algebra are linear sized types in \CIChatl~\citep{cic-hat-l},
which adds constant multiplication to a sized CoC with naturals and streams;
and well-founded sized types in \CIChatsub~\citep{wellfounded},
which changes the premise type checking the \cofixpoint body in Rules \refnorule{fix} and \refnorule{cofix}
to the recursive reference having \emph{any} smaller size according to the subsizing rules,
rather than the direct predecessor.
All three include size inference algorithms similar to that of \CIChat.

There are prototype implementations of \CIChatminus~\citep{cicminus} and \CIChatsub~\citep{cic-wf}.
It appears that there were also plans to implement sized types in Coq by \citet{coq-hat}, but seem to be abandoned.

\subsection{Past Work in Detail}

Past work \CIChat, \CIChatminus, and \CChatomega add sized types to CIC with the explicit philosophy of requiring no size annotations:
a user would write bare CIC code, and the type checker would have the simultaneous task of synthesizing and checking types,
while also inferring all the size annotations.
However, Coq's core calculus extends quite a bit beyond merely CIC,
and the presentation of various analogous features differ subtly but nontrivially.
The goal of \lang is to bring sized types in CIC a few steps closer to Coq,
while keeping with the original philosophy.
In the process of conforming to Coq's calculus, to minimize the changes required to it so that a prototype implementation is
viable, we must also discard some features from past work.

\subsubsection{Cumulativity}\label{sec:overview:comparison}

\CIChat and \CIChatminus are extensions of CIC, with dependent functions, a universe hierarchy, inductive definitions, case expressions, and fixpoints.
\CChatomega dually has coinductive streams and cofixpoints instead.
\CIChat and \CIChatminus differ in that \CIChat includes a size inference algorithm but no proof of strong normalization,
while \CIChatminus is proven to be strongly normalizing, with no size inference algorithm explicitly given.
\CIChatminus also restricts where size variables may appear in terms.
Since \lang doesn't have such restrictions, it can be thought of as an extension of \CIChat combined with \CChatomega,
featuring sized (mutual) \coinductive types and (mutual) \cofixpoints,
and further adding universe cumulativity, which is an existing feature in Coq.
As noted in \autoref{sec:metatheory}, cumulativity and impredicativity complicate the set-theoretic model by \citet{cic-hat-minus}.

\subsubsection{Definitions}

Coq's core calculus contains two kinds of variables:%
\footnote{It also has a third type of variable for section-level bindings;
this is beyond the scope of \lang.}
one for local bindings from functions, function types, and let expressions,
and one for global bindings from vernacular declarations such as \coqinline{Definition} and \coqinline{Axiom} (which we call \textit{constants}).
\lang adds let expressions and global declarations to \CIChat,
with separate local and global environments,
and definitions in the environments in the style of a PTS with definitions~\citep{PTSD}.

Global definitions and let expressions let us define aliases for types for concision and organization of code,
which necessitates a form of size polymorphism if we want the aliases to behave as we expect.
For instance, if we globally define $\Defn{N}{\Type{}}{\Nat^\upsilon}$,
and later want to define an addition function with type $N \rightarrow N \rightarrow N$,
it would \emph{not} be correct to perform the na\"ive substitution to get $\Nat^\upsilon \rightarrow \Nat^\upsilon \rightarrow \Nat^\upsilon$:
addition intuitively does not always return something of the same size.

What we want instead is to allow a different size for each use of $N$,
so that the above type reduces to $\Nat^{\upsilon_1} \rightarrow \Nat^{\upsilon_2} \rightarrow \Nat^{\upsilon_3}$.
This means $N$ must be polymorphic in the sizes involved in its definition,
the same kind of rank-1 or prenex polymorphism in ML-style let polymorphism for type variables.
To retain backward compatibility, there is no explicit size quantification or application ---
every definition and let binding is implicitly quantified over \emph{all} size variables involved.
The corresponding notion of size instantiation comes in the form of size substitution annotations on variables and constants, so that $N^{\set{\upsilon \mapsto s}}$ for instance reduces to $\Nat^s$.
% These and all other size annotations are inferrable, as detailed in \autoref{sec:algorithm}.

Having definitions and annotated variables and constants also means we need to now allow sizes to appear
not only in the bodies of let expressions but also in the bodies of functions and in the branches of case expressions,
in contrast to the restrictions of \CIChatminus.

\iffalse
\paragraph*{} Recall that the inference algorithm will return a size constraint set.
To handle inference of global declarations, after the size inference of one declaration,
we now have two options:
pass the resulting constraints along to inference of subsequent declarations,
or ``solve'' the constraints by reassigning size variables with sizes that satisfy those constraints.
Global definitions in Coq are independent of one another in terms of type checking,
so we choose the second option for its modularity:
constraints derived from previous definitions should not interfere with the size inference of the current definition.
However, solving constraints introduces a nontrivial overhead,
so we stick with the first option for local definitions.

Unfortunately, even when only solving constraints for global declarations,
our implementation of the size inference algorithm in the Coq codebase takes a rather significant performance hit.
This is in part due to the proliferation of sizes in which the definitions are polymorphic,
as the worst-case time complexity of solving is at least quadratic in the number of size variables,
and every usage of each definition requires fresh size variables for it.
We analyze the performance of our implementation and discuss possible solutions in \autoref{sec:impl}.
\fi

\subsubsection{Polarities}

\Coinductive definitions in \CIChatminus are also annotated with polarities for each of its parameters to augment the subtyping relation.
For example, if the type parameter of the usual $\List$ type is given a positive polarity,
then $\app{\List^r}{\Nat^s} \leq \app{\List^r}{\Nat^{\hat{s}}}$ holds
because $\Nat^s \leq \Nat^{\hat{s}}$ holds,
which in turn holds because $\hat{s}$ is a larger size than $s$.
Similarly, a negative polarity reverses the subtyping relation,
while an invariant polarity induces no subtyping relation from the parameters at all.
It is not known whether these polarity annotations are inferrable from the \coinductive definitions alone,
so again in the name of backward compatibility, \lang doesn't have these annotations,
and treats all parameters as invariant.
This aligns with Coq's current behaviour, where \coqinline{list Set} is not a subtype of \coqinline{list Type}
despite the presence of cumulativity where \coqinline{Set} is a subtype of \coqinline{Type}.

Unfortunately, the invariance of parameters and subtyping of sized \coinductive types interferes with nested \coinductive types,
where the type itself may appear as a parameter to another type in the type of its constructors.
Subject reduction is violated: it becomes possible to have a well-typed term that becomes no longer well typed after a reduction step,
as demonstrated in in \autoref{sec:metatheory:sub-red}.
The approach \lang takes is to disallow nested \coinductives, removing them from \CIChat.

\subsubsection{Implementation}

% Should the implementation just have a different name? Coq-hat-star?
Whereas \lang can be seen as an extension of \CIChat and \CChatomega,
its implementation is an extension of Coq:
all features of Coq orthogonal to sized types remain untouched,
such as universe polymorphism, strict $\Prop$, various primitives, modules, and so on.
The implementation also retains Coq's nested \coinductives,
especially as it doesn't appear possible for size inference to produce the kind of annotations that break subject reduction.

\subsection{Higher-Order Sized Types}

For the purposes of size inferrability from unannotated code,
the type systems from \lambdahat up to \CIChat and its variations treat sizes as merely annotations
and feature only what can be considered as prenex size polymorphism.
On the other hand, \citet{abel-diss} introduces \Fomegahat, a sized type system for System F$_\omega$ that treats size as a kind,
which therefore allows for higher-order size polymorphism via explicit quantification.
While \Fomegahat subsumes \Fhat and uses the same size algebra,
it uses recursive and corecursive type constructors ($\mu$- and $\nu$-types) rather than inductive (and coinductive) type definitions.

Higher-order sized types of the same flavour are implemented in a dependently-typed setting in MiniAgda~\citep{miniagda}.
To avoid inconsistencies introduced by the interplay between sized types and pattern matching,
it also introduces bounded size patterns \mbox{$\upsilon_1 < \upsilon_2$}.
\citet{flationary} expands upon the theoretical side with bounded size quantification \mbox{$\forall \upsilon \mathrel{<} s\mathpunct{.} t$} and well-founded recursion on sizes,
which are also implemented in MiniAgda.
\citet{f-omega-cop} combine well-founded sized types and copatterns for System F$_\omega$ with \corecursive type constructors in \Fomegacop (which was cited as the inspiration for \CIChatsub).

\cite{nbe} prove normalization of a higher-order sized dependent type theory with naturals, but without bounded size quantification.
To our knowledge, this is the only formalization of higher-order sized dependent types in the literature.
Additionally, they also add a notion of \emph{size irrelevance} so that more terms are convertible when their sizes don't matter;
this is an orthogonal feature that can be added to \lang as well,
since the issues size irrelevance aims to resolve can also arise here.

Sized types with higher-order bounded size quantification are implemented in Agda%
\footnote{For Agda's documentation on sized types, see \url{https://agda.readthedocs.io/en/latest/language/sized-types.html}.};
however, it is known to be inconsistent%
\footnote{For a detailed discussion on the issue, see \url{https://github.com/agda/agda/issues/2820}.}.
In short, it is possible to express the well-foundedness of sizes within Agda,
but the infinite size $\infty$ itself is \emph{not} well founded,
as $\infty + 1 = \infty$ and $\infty < \infty$ hold,
making it possible to derive a contradiction.

%%% Local Variables:
%%% TeX-master: "../main.tex"
%%% TeX-engine: default
%%% End:
