\begin{abstract}
Contemporary proof assistants such as Coq require that recursive functions be terminating and corecursive functions be productive to maintain logical consistency of their type theories,
and some ensure these properties using syntactic checks.
However, being syntactic, they are inherently delicate and restrictive,
preventing users from easily writing obviously terminating or productive functions at their whim.

Meanwhile, there exist many \emph{sized type theories} that perform type-based termination and productivity checking,
including theories based on the Calculus of (Co)Inductive Constructions (CIC), the core calculus underlying Coq.
These theories are more robust and compositional in comparison.
So why haven't they been adapted to Coq?

In this paper, we venture to answer this question with \lang, a sized type theory based on CIC.
It extends past work on sized types in CIC with additional Coq features such as global and local definitions.
We also present a corresponding size inference algorithm and implement it within Coq's kernel;
for maximal backward compatibility with existing Coq developments,
it requires no additional annotations from the user.

In our evaluation of the implementation, we find a severe performance degredation when compiling parts of the Coq standard library, inherent to the algorithm itself.
We conclude that if we wish to maintain backward compatibility,
using size inference as a replacement for syntactic checking is wildly impractical in terms of performance.

\iffalse
Termination of recursive functions and productivity of corecursive functions are important for maintaining logical consistency in proof assistants.
However, contemporary proof assistants, such as Coq, rely on fragile syntactic criteria that prevent users from easily writing some obviously terminating or productive functions.
This is troublesome, as there exist theories for type-based termination and productivity checking.

In this paper, we present a design and implementation of sized type checking and inference for Coq.
We extend past work on sized types for the Calculus of (Co)Inductive Constructions (CIC) to support definitions, and extend the sized type inference algorithm to support completely unannotated CIC terms.
This allows our design to maintain complete backward compatibility with existing Coq developments.
We provide an implementation that extends the Coq kernel with optional support for sized types.
\fi
\end{abstract}

%%% Local Variables:
%%% TeX-master: "../main.tex"
%%% TeX-engine: default
%%% End:
