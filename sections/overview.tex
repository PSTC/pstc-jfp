\section{Overview}\label{sec:overview}

Past work \CIChat, \CIChatminus, and \CChatomega add sized types to CIC with the explicit philosophy of requiring no size annotations:
a user would write bare CIC code, and the type checker would have the simultaneous task of synthesizing and checking types,
while also inferring all the size annotations.
However, Coq's core calculus extends quite a bit beyond merely CIC,
and the presentation of various analogous features differ subtly but nontrivially.
The goal of \lang is to bring sized types in CIC a few steps closer to Coq,
while keeping with the original philosophy.
In the process of conforming to Coq's calculus, to minimize the changes required to it so that a prototype implementation is
viable---for Coq's codebase itself is old, large, and intricate---we
must also discard some features from past work.
In this section, we discuss what \lang has added or removed relative to past work and to Coq,
and their implications on the metatheory and the implementation.
These features and their presence or absence in the relevant works are summarized in \autoref{table:CICs}.

\subsection{Comparison with Other Work}\label{sec:overview:comparison}

\CIChat and \CIChatminus are extensions of CIC, with dependent functions, a universe hierarchy, inductive definitions, case expressions, and fixpoints.
\CChatomega dually has coinductive streams and cofixpoints instead.
In terms of sized types, they add size expressions as annotations to \coinductive types,
and a \cofixpoint $f$ is well typed if all \corecursive calls in its body occurs on an argument of a smaller sizes,
as we have seen with the simplified typing rule in \autoref{sec:intro}.
Similarly to the example with the naturals,
the \corecursive arguments of a constructor must have a smaller size
than the fully-applied constructor itself.

Constructors construct constructions of a larger size than their \corecursive functions.

\CIChat and \CIChatminus differ in that \CIChat includes a size inference algorithm but no proof of strong normalization,
while \CIChatminus is proven to be strongly normalizing, with no size inference algorithm explicitly given.
\CIChatminus also restricts where size variables may appear in terms.
Since \lang doesn't have such restrictions, it can be thought of as an extension of \CIChat combined with \CChatomega,
featuring sized (mutual) \coinductive types and (mutual) \cofixpoints,
and further adding universe cumulativity, which is an existing feature in Coq.
Cumulativity and impredicativity complicate the set-theoretic model by \citet{cic-hat-minus}, but these issues are known: see, for instance, \citet{not-so-simple-cc}.

The size inference algorithms for \CIChat and \lang both take terms without size annotations,
which are in essence plain CIC terms, along with a local environment
and a set of size constraints for the environment,
and return an annotated term, its annotated size, and a new set of size constraints that must be satisfied.
Because \lang includes global definitions, its algorithm also takes a global environment.

\subsection{Definitions}

Coq's core calculus contains two kinds of variables:%
\footnote{It also has a third type of variable for section-level bindings;
this is beyond the scope of \lang.}
one for local bindings from functions, function types, and let expressions,
and one for global bindings from vernacular declarations such as \coqinline{Definition} and \coqinline{Axiom} (which we call \textit{constants}).
\lang adds let expressions and global declarations to \CIChat,
with separate local and global environments,
and definitions in the environments in the style of a PTS with definitions~\citep{PTSD}.

Global definitions and let expressions let us define aliases for types for concision and organization of code,
which necessitates a form of size polymorphism if we want the aliases to behave as we expect.
For instance, if we globally define $\Defn{N}{\Type{}}{\Nat^\upsilon}$,
and later want to define an addition function with type $N \rightarrow N \rightarrow N$,
it would \emph{not} be correct to perform the na\"ive substitution to get $\Nat^\upsilon \rightarrow \Nat^\upsilon \rightarrow \Nat^\upsilon$:
addition intuitively does not always return something of the same size.

What we want instead is to allow a different size for each use of $N$,
so that the above type reduces to $\Nat^{\upsilon_1} \rightarrow \Nat^{\upsilon_2} \rightarrow \Nat^{\upsilon_3}$.
This means $N$ must be polymorphic in the sizes involved in its definition,
the same kind of rank-1 or prenex polymorphism in ML-style let polymorphism for type variables.
To retain backward compatibility, there is no explicit size quantification or application ---
every definition and let binding is implicitly quantified over \emph{all} size variables involved.
The corresponding notion of size instantiation comes in the form of size substitution annotations on variables and constants, so that $N^{\set{\upsilon \mapsto s}}$ for instance reduces to $\Nat^s$.
These and all other size annotations are inferrable, as detailed in \autoref{sec:algorithm}.

Having definitions and annotated variables and constants also means we need to now allow sizes to appear
not only in the bodies of let expressions but also in the bodies of functions and in the branches of case expressions,
in contrast to the restrictions of \CIChatminus.

\paragraph*{} Recall that the inference algorithm will return a size constraint set.
To handle inference of global declarations, after the size inference of one declaration,
we now have two options:
pass the resulting constraints along to inference of subsequent declarations,
or ``solve'' the constraints by reassigning size variables with sizes that satisfy those constraints.
Global definitions in Coq are independent of one another in terms of type checking,
so we choose the second option for its modularity:
constraints derived from previous definitions should not interfere with the size inference of the current definition.
However, solving constraints introduces a nontrivial overhead,
so we stick with the first option for local definitions.

Unfortunately, even when only solving constraints for global declarations,
our implementation of the size inference algorithm in the Coq codebase takes a rather significant performance hit.
This is in part due to the proliferation of sizes in which the definitions are polymorphic,
as the worst-case time complexity of solving is at least quadratic in the number of size variables,
and every usage of each definition requires fresh size variables for it.
We analyze the performance of our implementation and discuss possible solutions in \autoref{sec:impl}.

\subsection{Polarities}

\Coinductive definitions in \CIChatminus are also annotated with polarities for each of its parameters to augment the subtyping relation.
For example, if the type parameter of the usual $\List$ type is given a positive polarity,
then $\app{\List^r}{\Nat^s} \leq \app{\List^r}{\Nat^{\hat{s}}}$ holds
because $\Nat^s \leq \Nat^{\hat{s}}$ holds,
which in turn holds because $\hat{s}$ is a larger size than $s$.
Similarly, a negative polarity reverses the subtyping relation,
while an invariant polarity induces no subtyping relation from the parameters at all.
It is not known whether these polarity annotations are inferrable from the \coinductive definitions alone,
so again in the name of backward compatibility, \lang doesn't have these annotations,
and treats all parameters as invariant.
This aligns with Coq's current behaviour, where \coqinline{list Set} is not a subtype of \coqinline{list Type}
despite the presence of cumulativity where \coqinline{Set} is a subtype of \coqinline{Type}.

Unfortunately, the invariance of parameters and subtyping of sized \coinductive types interferes with nested \coinductive types,
where the type itself may appear as a parameter to another type in the type of its constructors.
Subject reduction is violated: it becomes possible to have a well-typed term that becomes no longer well typed after a reduction step.
We give an example in \autoref{sec:metatheory:sub-red} and discuss some solutions,
but the approach \lang takes is to disallow nested \coinductives,
removing them from \CIChat.

\iffalse
\subsection{Position Size Annotations}

Although users aren't required to indicate the recursive argument when writing fixpoints in Coq,
they are indicated in the core calculus because Coq's type checker will attempt to determine which one it is when not explicitly indicated.
This tells the guard predicate which argument should be checked for structural smallerness.
Similarly, \lang has position annotations to indicate which argument is the recursive argument and whether the \cofixpoint preserves sizes or not,
but they are entirely determined during size inference.
This augments the inference algorithm for \CIChat,
which assumes that the annotations are already present.

Note that while inference begins with terms whose position annotations are absent,
we do assume that the recursive arguments are already indicated,
separating what Coq already does from what there is to do to obtain sized types.
In terms of the implementation, one downside of this approach is that not as many programs that \emph{could} pass type checking \emph{do},
due to the existing architecture of the Coq type checker.
We discuss this further in \autoref{sec:impl:recind}.
\fi

\subsection{Implementation}

% Should the implementation just have a different name? Coq-hat-star?
Whereas \lang can be seen as an extension of \CIChat and \CChatomega,
its implementation is an extension of Coq:
all features of Coq orthogonal to sized types remain untouched,
such as universe polymorphism, strict $\Prop$, various primitives, modules, and so on.
The implementation also retains Coq's nested \coinductives,
especially as it doesn't appear possible for size inference to produce the kind of annotations that break subject reduction.

%%% Local Variables:
%%% TeX-master: "../main.tex"
%%% TeX-engine: default
%%% End:
